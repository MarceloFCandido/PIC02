\begin{appendices}
\chapter{Códigos seriais}
\section{Códigos seriais falsos}
	\subsection{Cálculo da função de um paraboloide}
	\lstinputlisting[label={code:parabolloid-serial}, language=C, caption={solver.cpp: programa que calcula um paraboloide.}]
	{../src/serial-code/fake-code/parabolloid/src/solver.cpp}
	
	\lstinputlisting[label={code:view-parabolloid-serial}, language=Python, caption={viewer.py: script que cria a imagem das 
		projeções de camadas do paraboloide em um plano.}]
	{../src/serial-code/fake-code/parabolloid/src/viewer.py}
	
	\subsection{Propagação de onda em uma dimensão}
	\lstinputlisting[label={code:fdm-1d-serial}, language=C, caption={solver.cpp: programa que calcula a propagação de 
		uma onda em uma dimensão.}]
	{../src/serial-code/fake-code/fdm-1d/src/solver.cpp}
	
	\lstinputlisting[label={code:view-fdm-1d-serial}, language=Python, caption={viewer.py: script que cria a imagem da
		propagação da onda no plano $v \times t$.}]
	{../src/serial-code/fake-code/fdm-1d/src/viewer.py}

\section{Propagação de ondas em duas dimensões}
	\subsection{Interface de linha de comando para o usuário}
	\lstinputlisting[label={code:cli-serial}, language=C, caption={cli-main.cpp: programa para gerar as receber 
	as entradas do usuário para o programa serial.}]
	{../src/serial-code/real-code/src/cli/cli-main.cpp}

	\subsection{Bibliotecas criadas}
	\lstinputlisting[label={code:header-wave-serial}, language=C, caption={\_2DWave.h: arquivo-cabeçalho para a classe 
	que representa a onda.}]
	{../src/serial-code/real-code/src/libs/_2DWave.h}
	
	\lstinputlisting[label={code:wave-serial}, language=C, caption={\_2DWave.h: arquivo fonte para a classe 
		que representa a onda.}]
	{../src/serial-code/real-code/src/libs/_2DWave.cpp}
	
	\lstinputlisting[label={code:header-interface-serial}, language=C, caption={interface.h: arquivo-cabeçalho para a classe 
		que representa as interfaces.}]
	{../src/serial-code/real-code/src/libs/interface.h}
	
	\lstinputlisting[label={code:interface-serial}, language=C, caption={interface.cpp: arquivo-cabeçalho para a classe 
		que representa as interfaces.}]
	{../src/serial-code/real-code/src/libs/interface.cpp}
	
	\lstinputlisting[label={code:header-velocidade-serial}, language=C, caption={velocity.h: arquivo-cabeçalho para a classe 
		que representa as velocidades das camadas.}]
	{../src/serial-code/real-code/src/libs/velocity.h}
	
	\lstinputlisting[label={code:velocidade-serial}, language=C, caption={velocity.cpp: arquivo fonte para a classe 
		que representa as velocidades das camadas.}]
	{../src/serial-code/real-code/src/libs/velocity.cpp}

	\lstinputlisting[label={code:header-util-serial}, language=C, caption={util.h: arquivo-cabeçalho para a definição das 
		bibliotecas de sistema e constantes a serem utilizadas.}]
	{../src/serial-code/real-code/src/libs/util.h}
	
	\subsection{Ferramentas}
	\lstinputlisting[label={code:tool-snapshots-serial}, language=Python, caption={plot\_snapshots.py: \textit{script} 
	para geração dos instantâneos (\textit{snapshots}) da propagação da onda.}]
	{../src/serial-code/real-code/src/tools/plot_snapshots.py}

	\lstinputlisting[label={code:tool-gen-velocity-serial}, language=C, 
	caption={gen-velocities-model.cpp: \textit{script} para geração do 
		modelo de velocidades.}]
	{../src/serial-code/real-code/src/vels/gen-velocities-model.cpp}

	\lstinputlisting[label={code:tool-velocity-serial}, language=Python, 
	caption={plot\_velocity.py: \textit{script} para geração da imagem do modelo de 
		velocidades criado.}]
	{../src/serial-code/real-code/src/tools/plot_velocity.py}

	\lstinputlisting[label={code:tool-tracer-serial}, language=Python, 
	caption={tracer.py: \textit{script} para geração dos traços resultantes da 
		simulação.}]
	{../src/serial-code/real-code/src/tools/tracer.py}
	
	\lstinputlisting[label={code:tool-rsf-serial}, language=Python, 
	caption={traces2rsf.py: \textit{script} para conversão dos traços resultantes da 
		simulação para o formato \textit{.rsf} usado pelo Madagascar.}]
	{../src/serial-code/real-code/src/tools/traces2rsf.py}
	
	\lstinputlisting[label={code:tool-plot-madagascar-serial}, language=bash, 
	caption={plot\_traces.sh: \textit{script} para geração da imagem dos traços pelo Madagascar.}]
	{../src/serial-code/real-code/src/tools/plot_traces.sh}
	
	\subsection{Código principal}
	\lstinputlisting[label={code:main-serial}, language=C, 
	caption={main-reflect-bound.cpp: código fonte que realiza os cálculos da propagação da onda.}]
	{../src/serial-code/real-code/src/main/main-reflect-bound.cpp}
	
\chapter{Códigos falsos sobre OpenMP}
\section{Cálculo da função de um paraboloide}
\lstinputlisting[label={code:parabolloid-openmp}, language=C, 
caption={solver.cpp: código fonte que realiza os cálculos da função de um paraboloide com o auxílio 
	da OpenMP.}]
{../src/openMP/fc-openmp-parabolloid/src/solver.cpp}

\lstinputlisting[label={code:view-parabolloid-openmp}, language=Python, 
caption={viewer.py: código fonte que gera a imagem resultante dos cálculos da função de um paraboloide.}]
{../src/openMP/fc-openmp-parabolloid/src/viewer.py}

\section{Propagação de onda em uma dimensão}
%\lstinputlisting[label={code:fdm-1d-openmp}, language=C, 
%caption={solver.cpp: código fonte que realiza os cálculos da propagação de uma onda 
%	em meio unidimensional com o auxílio da OpenMP.}]
%{../src/openMP/fc-openmp-1d-fdm/src/solver.cpp}
%
%\lstinputlisting[label={code:view-parabolloid-openmp}, language=Python, 
%caption={viewer.py: código fonte que gera a imagem resultante dos cálculos da 
%	propagação da onda em meio unidimensional.}]
%{../src/openMP/fc-openmp-1d-fdm/src/viewer.py}
	
\chapter{Códigos falsos sobre Pthreads}
\section{Cálculo da função de um paraboloide}
\lstinputlisting[label={code:parabolloid-pthreads}, language=C, 
caption={solver.cpp: código fonte que realiza os cálculos da função de um paraboloide com o auxílio 
	da Pthreads.}]
{../src/pthreads/fc-pthreads-parabolloid/src/solver.cpp}

\lstinputlisting[label={code:view-parabolloid-pthreads}, language=Python, 
caption={viewer.py: código fonte que gera a imagem resultante dos cálculos da função de um paraboloide.}]
{../src/pthreads/fc-pthreads-parabolloid/src/viewer.py}

\section{Propagação de onda em uma dimensão}
\lstinputlisting[label={code:fdm-1d-pthreads}, language=C, 
caption={solver.cpp: código fonte que realiza os cálculos da propagação de uma onda 
	em meio unidimensional com o auxílio da Pthreads.}]
{../src/pthreads/fc-pthreads-1d-fdm/src/solver.cpp}

\lstinputlisting[label={code:view-parabolloid-pthreads}, language=Python, 
caption={viewer.py: código fonte que gera a imagem resultante dos cálculos da 
	propagação da onda em meio unidimensional.}]
{../src/pthreads/fc-pthreads-1d-fdm/src/viewer.py}

\chapter{Códigos falsos sobre MPI}
\section{Cálculo da função de um paraboloide}
\lstinputlisting[label={code:parabolloid-mpi}, language=C, 
caption={solver.cpp: código fonte que realiza os cálculos da função de um paraboloide com o auxílio 
	da MPI.}]
{../src/mpi/fc-mpi-parabolloid/src/solver.cpp}

\lstinputlisting[label={code:view-parabolloid-mpi}, language=Python, 
caption={viewer.py: código fonte que gera a imagem resultante dos cálculos da função de um paraboloide.}]
{../src/mpi/fc-mpi-parabolloid/src/viewer.py}

\chapter{Códigos finais}
\section{Códigos serial final falso}
\subsection{Propagação de onda em uma dimensão}
\lstinputlisting[label={code:fdm-1d-parallel}, language=C, caption={solver.cpp: resolve a equação da onda unidimensional 
com o auxílio das APIs Pthreads e MPI}]
{../src/final/fake-code/src/solver.cpp}

\lstinputlisting[label={code:view-fdm-1d-parallel}, language=Python, caption={viewer.py: script que cria a imagem da
	propagação da onda no plano $v \times t$.}]
{../src/final/fake-code/src/viewer.py}

\section{Propagação de ondas em duas dimensões}
\subsection{Código principal}
\lstinputlisting[label={code:fdm-2d-parallel}, language=C, 
caption={main-reflect-bound.cpp: código fonte que realiza os cálculos da propagação da onda bidimensional com 
	o auxílio das APIs Pthreads e MPI.}]
{../src/serial-code/real-code/src/main/main-reflect-bound.cpp}

\end{appendices}