
%%%%%%%%%%%%%%%%%%%%%%%%%%%%%%%%%%%%%%%%%
% University Assignment Title Page
% LaTeX Template
% Version 1.0 (27/12/12)
%
%
% Original author:
% WikiBooks (http://en.wikibooks.org/wiki/LaTeX/Title_Creation)
%
% License:
% CC BY-NC-SA 3.0 (http://creativecommons.org/licenses/by-nc-sa/3.0/)
%
% 1) Copy/paste everything between \begin{document} and \end{document}
% starting at \begin{titlepage} and paste this into another LaTeX file where you
% want your title page.
% OR
% 2) Remove everything outside the \begin{titlepage} and \end{titlepage} and
% move this file to the same directory as the LaTeX file you wish to add it to.
% Then add \input{./title_page_1.tex} to your LaTeX file where you want your
% title page.
%
%%%%%%%%%%%%%%%%%%%%%%%%%%%%%%%%%%%%%%%%%

\title{Uma introdução a Computação de Alto Desempenho}
%-------------------------------------------------------------------------------
%   PACKAGES AND OTHER DOCUMENT CONFIGURATIONS
%-------------------------------------------------------------------------------

\documentclass[12pt, a4paper]{report}
% Importando pacotes básicos
% Codificação
\usepackage[utf8]{inputenc}
% Linguagem
\usepackage[brazil]{babel}
% Fonte
\usepackage[T1]{fontenc}
% Determinando as margens do documento
\usepackage[left=3cm, right=2cm, bottom=2cm, top=3cm]{geometry}
% Possibilita o uso de LINKs e URLs ao longo do documento
\usepackage[pdftex, hidelinks]{hyperref}
% possibilita o uso do simbolo de graus
\usepackage{gensymb}

% Pacotes que não lembro a funcionalidade
\usepackage{setspace}
\usepackage{graphicx} % Uso de figuras (ACHO)
\usepackage{float}
\usepackage{mathptmx}
\usepackage{amsmath}

% Permitir quebra de página nos ambientes de equações
\allowdisplaybreaks

% Acho que usei isso para colocar figuras em tabelas
\usepackage{subfig}

% Anexar PDFs no LaTeX
\usepackage[final]{pdfpages}

% Para colocar apêndices no LaTeX
\usepackage[toc, page]{appendix}

% Insercao de codigos no LaTeX
\usepackage{listings}
\lstset{language=Python}    % Setando a linguagem padrao

\usepackage{color}

\definecolor{codegreen}{rgb}{0,0.6,0}
\definecolor{codegray}{rgb}{0.5,0.5,0.5}
\definecolor{codepurple}{rgb}{0.58,0,0.82}
\definecolor{backcolour}{rgb}{0.95,0.95,0.92}

% Definindo estilo de exibição de código
\lstdefinestyle{mystyle}{
	backgroundcolor=\color{backcolour},
	commentstyle=\color{codegreen},
	keywordstyle=\color{magenta},
	numberstyle=\tiny\color{codegray},
	stringstyle=\color{codepurple},
	basicstyle=\footnotesize,
	breakatwhitespace=false,
	breaklines=true,
	captionpos=b,
	keepspaces=true,
	numbers=left,
	numbersep=5pt,
	showspaces=false,
	showstringspaces=false,
	showtabs=false,
	tabsize=4,
	basicstyle=\scriptsize
}

% Para adicionar notas e TODO's
\usepackage{todonotes}

% Para colocar margem da na legenda de um figure/table
% \usepackage[margincaption,outercaption,ragged,wide]{sidecap}
% \sidecaptionvpos{figure}{t}
\usepackage[figurename=Fig.]{caption}
%\DeclareCaptionStyle{figstyle}
%  [format=plain,margin=0pt,justification=centering]
%  {format=hang,calcmargin={0pt,\widthof{\captionfont\captionlabelfont\figurename
%  ~\thefigure: }},
%   font=small,labelfont=bf}
%\captionsetup[figure]{style=figstyle}

% Indenta o primeiro parágrafo das seções
\usepackage{indentfirst}

\usepackage[splitrule]{footmisc}

% Pacote para o uso do biblatex como bibliografia
\usepackage[
    backend=biber,
    style=alphabetic,
    sorting=ynt
]{biblatex}
% Definindo o arquivo .bib de bibliografia
\addbibresource{BIB.bib}

% Translating mathematical function names to Portuguese
\let\sin\relax        \DeclareMathOperator{\sin}{sen}
\let\arcsin\relax     \DeclareMathOperator{\arcsin}{arcsen}

\begin{document}

	\begin{titlepage}

	\newcommand{\HRule}{\rule{\linewidth}{0.5mm}} % Defines a new command for the horizontal lines, change thickness here

	\center % Center everything on the page

	%----------------------------------------------------------------------------------------
	%   HEADING SECTIONS
	%----------------------------------------------------------------------------------------

	\textsc{\LARGE Centro Federal de Educação Tecnológica de Minas Gerais}\\[3.5cm] % Name of your university/college
	\textsc{\Large Engenharia de Computação}\\[3.5cm] % Major heading such as course name

	%----------------------------------------------------------------------------------------
	%   TITLE SECTION
	%----------------------------------------------------------------------------------------

	\HRule \\[0.4cm]
	{ \huge \bfseries Paralelização de Métodos Numéricos para Resolver
	Equações Diferenciais Parciais}\\[0.2cm] % Title of your document
	\HRule \\[2.5cm]

	%----------------------------------------------------------------------------------------
	%   AUTHOR SECTION
	%----------------------------------------------------------------------------------------

	\begin{minipage}{0.4\textwidth}
		\begin{flushleft} \large
			\emph{Orientando:}\\ Marcelo Lopes de Macedo \textsc{Ferreira Cândido} % Your name
		\end{flushleft}
	\end{minipage}
	~
	\begin{minipage}{0.4\textwidth}
		\begin{flushright} \large
			\emph{Orientador:} \\
			Prof. Dr. Luis Alberto \textsc{D'Afonseca} % Supervisor's Name
		\end{flushright}
	\end{minipage}\\[7cm]

	% If you don't want a supervisor, uncomment the two lines below and remove the section above
	%\Large \emph{Author:}\\
	%John \textsc{Smith}\\[3cm] % Your name

	%----------------------------------------------------------------------------------------
	%   DATE SECTION
	%----------------------------------------------------------------------------------------

	{\large BELO HORIZONTE\\\today}\\[1cm] % Date, change the \today to a set date if you want to be precise

	%----------------------------------------------------------------------------------------
	%   LOGO SECTION
	%----------------------------------------------------------------------------------------

	%\includegraphics{logo.png}\\[1cm] % Include a department/university logo - this will require the graphicx package

	%----------------------------------------------------------------------------------------

	\vfill % Fill the rest of the page with whitespace

	\end{titlepage}

    \lstset{style=mystyle}

    \tableofcontents

    \chapter{Introdução}

\label{chp1:introduction}
O uso de computadores para a comunicação global, bem como para o avanço da ciência e da tecnologia tornou-se indispensável. Seja na meteorologia, no combate a doenças, nas áreas militar, espacial, de construção civil e mineração e até mesmo nos esportes e cinema, os computadores são ferramentas essenciais para a construção e obtenção de novas ideias. Além disso, pode-se dizer que dentro da necessidade computacional, encontra-se outra: \textbf{velocidade de processamento}, já que os problemas a serem resolvidos em áreas como as citadas são urgentes e também podem ter diferentes escalas e complexidades. Contudo, aumentar a velocidade dos processadores tornou-se inviável. Isso levou à uma alternativa: \textit{chips} de processamento com mais de um processador, o que trouxe novos desafios aos desenvolvedores de \textit{software}.

Esse projeto de iniciação científica visou apresentar um estudo sobre paradigmas de processamento paralelo (em mais de um núcleo de processamento e em mais de uma máquina) utilizando uma prática no ramo de sísmica (a aquisição sísmica) como exemplo. Será modelado um programa paralelo a partir de um serial que simulará a propagação de uma onda bidimensional (utilizando a equação da onda e o Método de Diferenças Finitas) em um meio não-homogêneo e a aquisição de dados sobre essa propagação.

O presente relatório primeiro explicará o que é uma aquisição sísmica e como ela pode ser modelada. Em seguida, discursará sobre conceitos computacionais e sobre paradigmas do processamento paralelo. Após isso, serão apresentadas algumas estruturas de código de terceiros que auxiliarão na produção dos códigos final e auxiliares. Por fim, se apresentará o processo de construção do código objetivo capaz de resolver o problema em várias máquinas ao mesmo tempo (ou seja, distribuidamente) e utilizando mais de um núcleo de processamento por máquina, quando disponível.

\section{Notas}
Primeiramente é necessário dizer que o presente projeto é a continuação de Cândido \cite{mfcandido2018}, no qual objetivava-se comparar dois métodos para simulação da propagação de ondas bidimensionais em meios não-homogêneos. Esses métodos eram: traçamento de raios e diferenças finitas.

Outra nota importante é que a maior parte do código desse projeto foi escrito em C++, com o auxílio da biblioteca \texttt{Armadillo} e das \gls{api}s \texttt{OpenMP}, \texttt{Pthreads} e \texttt{MPI}. A linguagem Python também foi utilizada para visualização de gráficos, com o auxílio das bibliotecas \texttt{numpy} e \texttt{matplotlib}. Outras porções de código, sendo, nesse caso, boa parte feitas pelo orientador; foram implementadas utilizando outras linguagens como \texttt{Shell Script}.

Por fim, o último detalhe a se salientar foi a necessidade da intervenção pontual do orientador em alguns trechos de código em que o orientando não conseguia prosseguir. 

    \chapter{Aquisições Sísmicas}

    \label{chp2}
    
    \todo[inline]{introduzir}

    \section{O Que São Aquisições Sísmicas e Como Modelá-las}

    \section{A Equação da Onda}

	Para que seja possível avaliar matematicamente um fenômeno ondulatório 
	produzido por uma fonte em um domínio é necessário se ter uma fórmula 
	matemática para o que se entende por onda. Tal fórmula, que nos servirá 
	durante todo esse trabalho, principalmente na parte do código é a 
	\textbf{equação da onda}, dada por 
	\begin{equation}
		% dtt = 1 / v^2 (dxx + dyy) + f(x, y, t)
		\dfrac{\partial^2 u}{\partial t^2} = 
			(\dfrac{1}{v^2})
			\dfrac{\partial^2 u}{\partial x^2} + 
			\dfrac{\partial^2 u}{\partial y^2} + f(x, y, t)
	\end{equation}
	onde $x$ e $y$ são variáveis espaciais e $t$, temporal. A constante (no caso
	desse trabalho) $v$ representa a velocidade da frente de onda. Trata-se de uma 
	\textbf{equação diferencial parcial hiperbólica} com solução analítica, mas que 
	pode ser resolvida de forma fácil numericamente, utilizando, por exemplo, o 
	\textbf{método de diferenças finitas}, a ser mais explicado na próxima seção.
	
	Caso o leitor se interesse por estudar ou revisar mais sobre Ondulatória, pode 
	conferir em Cândido \cite{mfcandido2018}.


    \section{O Método de Diferenças Finitas (MDF)}

	Uma equação diferencial parcial, que geralmente é considerada em um domínio 
	contínuo, pode ser discretizada. Isso é feito para que a equação possa ser 
	representada e resolvida computacionalmente. 
	
	No caso do método de diferenças	finitas para esse trabalho, basta 
	transcrever cada termo da equação para o equivalente na fórmula de 
	diferenças finitas para derivações de segundo grau, sobre a qual podemos 
	ver um exemplo na Tabela \ref{tab:stencil1D} para o caso da parte espacial em $x$ da equação. 
	\begin{table}[H]
		\caption{Fórmula de Diferenças Finitas para cálculo de derivada de segunda ordem, aqui representando a derivada na dimensão espacial $x$}
		\centering
		\begin{tabular}{c|c}
			\hline
			$\dfrac{\partial^2 u}{\partial x^2}$ & $\dfrac{u_{i-1,j,k} - 2u_{i,j,k} + 
				u_{i+1,j,k}}{\Delta x^2}$ \\ \hline
		\end{tabular}
		\label{tab:stencil1D}
	\end{table}
	
	
	Para entender o que 
	significa o índice $i$ visto na tabela acima é bom se seguir uma analogia: 
	suponhamos um \textit{array} de tamanho maior que três. A posição $i$ seria 
	qualquer posição intermediária, $i-1$ a antecessora e a $i+1$ sucessora. Tal 
	estrutura é chamada de \textit{stencil}. O mesmo vale para os índices
	$j$ e $k$. Podemos ver uma alegoria dessa 
	analogia na Figura \ref{fig:stencil1}.
	
	\todo[inline]{Inserir uma imagem para o stencil 1D}
%	\begin{figure}
%		\centering
%       \label{fig:stencil1}
%		\includegraphics[scale=.5, width=\textwidth]{imagefile}
%		\caption{\textit{Stencil} unidimensional}
%	\end{figure}
	
	No caso desse trabalho, para a simulação da propagação de ondas em um 
	meio bidimensional ao longo do tempo, teremos que usar três 
	\textit{stencils} (os dois restantes podem ser vistos na Tabela 
	\ref{tab:stencils3D}), um para cada dimensão espacial ou temporal. Para tal, 
	podemos utilizar outra analogia: um \textit{array} tridimensional, onde cada
	plano de posições no espaço é um instante no tempo. Tal analogia é representada 
	na Figura \ref{fig:stencil3D}.
	
	\begin{table}[H]
		\centering
		\caption{Demais fórmulas de Diferenças Finitas para as dimensões
			espacial $y$ e temporal $t$}
		\begin{tabular}{c|c}
			\hline
			$\dfrac{\partial^2 u}{\partial y^2}$ & $\dfrac{u_{i,j-1,k} - 2u_{i,j,k} + 
				u_{i,j+1,k}}{\Delta y^2}$ \\ \hline
			$\dfrac{\partial^2 u}{\partial t^2}$ & $\dfrac{u_{i,j,k-1} - 2u_{i,j,k} + 
				u_{i,j,k+1}}{\Delta t^2}$ \\ \hline
		\end{tabular}
		\label{tab:stencils3D}
	\end{table}
	
	%	\begin{figure}[H]
	%		\centering
	%       \label{fig:stencil3D}
	%		\includegraphics[scale=.5, width=\textwidth]{imagefile}
	%		\caption{\textit{Stencil} tridimensional}
	%	\end{figure}
	
	Contudo, até então, não falamos sobre o método de Diferenças Finitas 
	em si. Trata-se de uma sequência de iterações que marcham em função de
	alguma variável. No caso desse trabalho, o avanço se dá no tempo. Essa marcha no tempo quer dizer que o valor para um ponto no espaço no próximo instante de tempo será calculado com base nos valores para pontos no espaço em instantes anteriores. Vamos ser explícitos. Temos que a equação \ref{eq:waveEq} traduzida nas fórmulas vistas nas tabelas 
	\ref{tab:stencil1D} e \ref{tab:stencils3D} se dá por
	\begin{equation}
		 \dfrac{u_{i,j,k-1} - 2u_{i,j,k} + u_{i,j,k+1}}{\Delta t^2} = \dfrac{1}{v^2}
			 \left(\dfrac{u_{i-1,j,k} - 2u_{i,j,k} + \\u_{i+1,j,k}}{\Delta x^2} + 
			 \dfrac{u_{i,j-1,k} - 2u_{i,j,k} + u_{i,j+1,k}}{\Delta y^2}\right) + f(x, y, t)
	\end{equation}
	onde $u_{k+1}$ é o ponto com valor a ser calculado para o próximo
	instante. Logo, ele precisa ser isolado, o que é mostrado na equação 
	\ref{eq:timeMarching}
	\begin{equation}
		\label{eq:timeMarching}
		u_{i, j,k+1} = \dfrac{\Delta t^2}{v^2}
		\left(\dfrac{u_{i-1,j,k} - 2u_{i,j,k} + u_{i+1,j,k}}{\Delta x^2} + 
		\dfrac{u_{i,j-1k} - 2u_{i,j,k} + u_{i,j+1,k}}{\Delta y^2}\right) + \Delta t^2f(x, y, t) - u_{i,j,k-1} + 2u_{i,j,k}
	\end{equation}
		\todo[inline]{buscar referências sobre o método na PIC01}



	
\chapter{A Arquitetura Computacional Atual e a Necessidade de Paralelização}

Para compreender a questão da paralelização envolvida nesse trabalho é
necessário entender de onde e porque ela veio. Para tal, é necessário
se apresentar as principais peças que constituem um computador moderno,
os problemas que surgiram no processo de evolução da \textbf{arquitetura
computacional} convencional (baseando-se na \gls{von-Neumann}) e como isso culminou no 
paralelismo \cite{LLNL:parcomp}.

Antes de iniciar esse processo, é também necessário se explicar o que se
entende por arquitetura computacional. Trata-se da área do conhecimento
que estuda a interface entre \textit{software} e \textit{hardware}, desde
o mais baixo nível, no qual o processador manipula as informações
(\gls{instr-machine} e dados) entregues a ele, para toda operação realizada no
computador. Após esse nível, tem-se as políticas de manipulação de dados nas memórias
cache (e seus níveis), de acesso aleatório (\acrshort{RAM}) e de 
armazenamento não-volátil (discos rígidos, por exemplo).
Por fim, chega-se à interação dos computadores com os demais periféricos que
por ventura estão nele conectados, realizando \textit{inputs} (entradas)
e/ou \textit{outputs} (saídas), também conhecidas pela abreviação "I/O",
como teclado, \textit{mouse}, monitor, etc \cite{Catsoulis}.

O presente capítulo abordará alguns componentes da arquitetura computacional 
(processador e memória \acrshort{RAM}) de uma forma básica, além de explicar por quais 
motivos a paralelização se tornou necessária e como ela se desenvolveu.

\section{O Processador}

\label{sec:processor}

Um processador consiste em um módulo de \textit{hardware} capaz de manipular 
\gls{instr-machine} armazenadas
em memória e produzir os resultados desejados através dessas instruções. Tais resultados
podem ser de cunho ou lógico-aritmético ou manipulação de dados, no geral. Tal módulo é
indispensável para o conceito de computadores como conhecemos hoje, de tal forma que, se
não fosse pela necessidade de memória para a armazenagem de dados, um processador poderia
ser a definição de um computador.

\section{A memória principal}

    \begin{itemize}
        \item introdução do que é uma memória;
        \item apresentação básica dos tipos de memória 
        (enfase na RAM);
        \item apresentação básica das propriedades de uma 
        memória RAM.
    \end{itemize}

\section{Conseguir mais em menos tempo}

	A computação sempre foi uma das principais ferramentas humanas para avanços 
	tecnológicos nos últimos séculos. Seja na engenharia, medicina, área 
	militar, biologia, química, sísmica e até no cinema, os computadores tem 
	sido utilizados em tarefas como mecânicas, estudos patológicos, 
	ataques/defesas nacionais, enovelamento de proteínas, dinâmica molecular, 
	monitoramento sísmico e animações tridimensionais.
	
	Todas essas áreas costumam requerer que os resultados obtidos 
	computacionalmente sejam gerados o mais rápido possível. Além disso, 
	muitas (senão quase todas) as \textbf{simulações científicas}\footnote{
		explicar o que seriam simulações científicas}
	envolvidas não podem ser executadas em tempo hábil com o tipo de processador
	visto na Seção \ref{sec:processor}. Visto 
	isso, os projetistas por trás dos projetos de processadores precisaram 
	implementar mecanismos nos modelos para explorá-los ao máximo. Nessa tarefa, 
	encontraram barreiras \todo{Ir à biblioteca}\cite{autor:intro-par-comp}.
    
    \subsection{A barreira do paralelismo a nível de instruções - \textit{ILP wall}}
    
    	Vimos na Seção \ref{sec:processor} um processador com a capacidade de 
    	executar uma instrução por ciclo de \textit{clock}. Com isso, a duração 
    	do ciclo devia ser a mesma da execução da instrução mais demorada. 
    	Na necessidade de se adquirir mais velocidade de processamento, 
    	os projetistas perceberam que podiam particionar as instruções, de modo 
    	a executar cada estágio resultante em um ciclo de \textit{clock}, 
    	deixando o resultado para a próxima partição a ser operada. Assim 
    	nasceu o processador \textbf{multiciclo}. Dessa forma, a duração 
    	do ciclo de \textit{clock} reduziu para a mesma do estágio mais 
    	demorado dentre as instruções.
    	
    	Contudo, a necessidade de se acelerar os processadores continuava. Em
    	seguida, os projetistas perceberam que as unidades funcionais dos 
    	processadores ficavam ociosas quando não se tratava do estágio de uma 
    	instrução em que elas eram utilizadas. Era então possível executar uma
    	instrução ao mesmo tempo que outra, desde que ambas se encontrassem, 
    	cada uma, em estágios $A$ e $B$, sendo $A$ o estágio da instrução mais 
    	antiga e $B$ o da mais nova,  com $B = A - 1$. 
    	
    	Ou seja, por exemplo, 
    	considere uma instrução $i$ liberada no tempo de \textit{clock} $1$, 
    	passando por seu primeiro estágio. No tempo $2$, ela estará em sua
    	segunda etapa e as unidades funcionais responsáveis pela primeira 
    	estariam ociosas, se não fosse pela inovação apresentada acima. Com 
    	esta, uma nova instrução $j$ é buscada ainda nesse tempo, tendo 
    	seu primeiro estágio executado. No tempo de ciclo de \textit{clock} 3,
    	a instrução $i$ passará para o seu terceiro estágio, enquanto a $j$
    	passará para o segundo e uma nova instrução poderá ser buscada. À essa 
    	inovação, foi dado o nome de \textbf{paralelismo a nível de instruções}, 
    	visto que se tem mais de uma instrução sendo executada ao mesmo tempo e 
    	uma pronta a cada ciclo de \textit{clock}.
    	
    	Em seguida, os projetistas
    
        \begin{itemize}
            \item Explicar os conceitos de superpipeline e superescalar;
            \item Explicar que, ao se estender muito um pipeline, temos problemas;
            \item Explicar os problemas da superescalaridade
        \end{itemize}
    
    \subsection{A barreira no gasto de energia dos processadores - \textit{Power wall}}
    
        \begin{itemize}
            \item Introduzir a lei de Moore;
            \item Explicar que houve evolução na taxa de clock ao longo do tempo;
            \item Explicar que essa evolução foi amortecida nos últimos anos devido 
            ao gasto energético e às altas temperaturas que os processadores alcançaram 
        \end{itemize}
    
    \subsection{A barreira da memória - \textit{Memory wall}}
    
	    \todo[inline]{Conferir se existe mesmo a memory wall}
	    \todo[inline]{Revisar o que é a memory wall e completar o itemize abaixo}
	    
	    \begin{itemize}
	    	\item 
	    \end{itemize}

\section{Paralelismo - a alternativa para se contornar as barreiras}

	\label{sec:parallelism}
    
    Visto a incapacidade de se transpor o alto gasto energético das 
    unidades de processamento, junto com as consequentes altas temperaturas,
    além da incapacidade de se aumentar o número de instruções prontas por 
    ciclo de \textit{clock}, precisamos de uma alternativa para se continuar 
    aumentando o poder de processamento dos computadores. 
    
    Tal alternativa foi então aumentar o número de unidades de processamento,
    seja em um \textit{chip} com mais de uma unidade (chamada núcleo, ou 
    \textbf{\textit{core}} em inglês) ou em um sistema com mais de uma 
    máquina, que por sua vez possui uma ou mais unidades de processamento. 
    Dessa forma, o número de tarefas concluídas por unidade de tempo aumentou,
    visto.
    A isso foi dado o nome de \textbf{paralelismo}.
    
    Existem muitos motivos para se utilizar o paralelismo. O mais importante é
    que o universo possui muitos processos paralelos, ou seja, ocorrendo ao 
    mesmo tempo, tais como mudanças climáticas, montagens de veículos e 
    aeronaves, tráfico em um pedágio e acessos a um site. Em consequência da 
    necessidade de se processar esses eventos por computadores, surgem outros motivos para a computação paralela:
    \begin{itemize}
    	\item poupar tempo e/ou dinheiro: existem problemas computacionais 
    	que são muito demorados, senão impossíveis, para se resolver
    	serialmente. Dessa forma, usando-se a computação serial, se paga mais,
    	visto o uso por mais tempo do poder computacional;
    	\item resolver problemas maiores e/ou mais complexos, visto o
    	processamento mais rápido e a possibilidade de se dividir o trabalho
    	mais máquinas;
    	\item prover concorrência: realizar várias tarefas simultaneamente
    	\cite{LLNL:parcomp}.
    \end{itemize}
    
    Quando se fala em paralelismo, é importante mostrar que existem diferentes
    disposições de memória nesse meio, o que veremos na subseção
    \ref{subsec:memory-architectures}. Além disso, existem diferentes
    paradigmas de paralelismo, sendo alguns brevemente explicados na subseção
    \ref{subsec:par-comp-models}. Há também outros assuntos a serem tratados 
    ao se projetar programas paralelos, sendo alguns abordados na subseção 
    \ref{subsec:parallel-design}. Fora isso, mais informações sobre 
    paralelismo podem ser encontradas em Barney \cite{LLNL:parcomp}.
    
    \subsection{Arquiteturas de memória na computação paralela}
    
	    \label{subsec:memory-architectures}
	    
	    Na computação paralela, existem diferentes formas pelas quais a 
	    memória é administrada e interpretada. Isso pode ser realizado fisica 
	    e/ou logicamente, principalmente de três formas:
        \begin{itemize}
            \item memória compartilhada: cada unidade de processamento de
            um sistema tem acesso à toda a memória, com um \textbf{espaço de
            endereçamento}\footnote{\textbf{espaço de endereçamento}, segundo
            a Wikipédia \cite{wiki:address}, é uma ``série de endereços
            discretos'' que, no caso, nomeiam as posições de memória} global;
            
            \item memória distribuída: cada unidade de processamento possui
            sua própria memória, mapeada apenas para ele e que pode ser 
            acessada pelos demais nós através de requisições feitas por meio
            de uma rede que os liga. Além disso, cada nó opera
            independentemente, visto que cada um tem sua própria memória.
            Dessa forma, as mudanças que um nó opera sobre sua própria 
            memória não afetam as dos demais nós;
            
            \item híbrida: literalmente a mescla de ambas, ou seja, cada 
            \textbf{nó}\footnote{\textbf{nó} é o nome dado na computação 
            paralela para cada unidade de processamento (ou conjunto dessas) 
            independente.} possui mais de uma unidade de processamento e sua 
	        própria memória, sendo também capaz de acessar as dos outros 
	        \cite{LLNL:parcomp}.
        \end{itemize}
        
    
    \subsection{Modelos da computação paralela}
    
		\label{subsec:par-comp-models}
		Os modelos de programação paralela 
    
        \begin{itemize}
            \item Shared Memory Model;
            \item Threads Model;
            \item Distributed Memory / Message Passing Model;
            \item Data Parallel Model;
            \item Hybrid Model
        \end{itemize}
    
    \subsection{Desenhando programas paralelos}
    
	    \label{subsec:parallel-design}
    
        \begin{itemize}
            \item explicar a diferença entre a paralelização automática e a 
            manual;
            \item explicar a necessidade de se entender o problema e o programa;
            \item explicar
            \begin{itemize}
            	\item particionamento;
            	\item sincronização.
            \end{itemize}

        \end{itemize}

    \chapter{Paralelismo em prática}
    
    \section{OpenMP}

A \acrshort{api} \acrfull{openmp} consiste em rotinas, 
variáveis de ambiente e diretivas de compilação reunidas em um 
modelo portável e escalável que serve como uma interface para 
desenvolvedores criarem aplicações paralelas com simplicidade e 
flexibilidade \cite{wiki:openmp}. Essa \acrshort{api} se encontra 
incluída no modelo de computação paralela de memória compartilhada, 
assim como a \acrshort{pthreads}.

\subsection{Diretiva Utilizada}
\begin{lstlisting}
#pragma omp parallel for
\end{lstlisting}
Considerada uma diretiva de compartilhamento de trabalho, essa 
ferramenta permite que qualquer laço de repetição do tipo \texttt{for} 
(nas linguagens C/C++, 
\lstinline[columns=fixed]{for(int i = 0; i < ALGUM_NUMERO; i++)// faz algo}, 
por exemplo) tenha seu número de iterações dividido entre $n$ \textit{\gls{threads}}, 
sendo $0 < n <= \text{número máximo de threads}$.
    
    \section{Pthreads}

	\begin{itemize}
		\item Do que se trata a Pthreads e onde ela está incluída;
		\item um programa hello world com Pthreads.
	\end{itemize}
    
    \section{MPI}

	\begin{itemize}
		\item Do que se trata a MPI e onde ela está incluída;
		\item um programa hello world com MPI.
	\end{itemize}
    

    \chapter{Do serial ao paralelo}

	\label{cap:implementation}
	
	Como já dito anteriormente, o objetivo desse trabalho é a construção 
	de um código paralelizado capaz de resolver a equação da onda utilizando
	o Método de Diferenças Finitas. O presente capítulo apresentará a 
	construção desse código, partindo do serial equivalente, passando pelos 
	estudos realizados com as \glspl{api} e concluindo com a construção do 
	código paralelo final.

    \section{A construção do código serial}

Como dito anteriormente na Seção \ref{subsec:parallel-design}, 
após se compreender o problema a ser programado em termos paralelos,
deve-se então criar o código serial que resolve o problema.
    
    \section{Primeiro contato do código com as \textit{threads} - OpenMP}

Essa seção pretende demonstrar os esforços realizados para se construir dois
códigos ``falsos'' que, como já dito, foram criados para facilitar a criação do código final.
Esses códigos são variações dos códigos ``falsos'' seriais, realizando as mesmas tarefas
(avaliação de função e cálculo da propagação de ondas em meio unidimensional), mas
com o auxílio da \gls{api} \acrshort{openmp}.

\subsection{Avaliação de uma função}

O Código \ref{code:parabolloid-openmp} utiliza a estrutura
\begin{lstlisting}
#pragma omp parallel for default(none) shared(A, x_b, y_b, x_ofst, y_ofst, x_points, y_points) private(i, j, x_i, y_i)
\end{lstlisting}
cuja versão mais simples foi descrita na Seção \ref{sec:openmp},  para aplicar o paralelismo no problema. 

A estrutura \texttt{default(none)} serve para explicitar que o tipo das variáveis (\texttt{shared} ou \texttt{private}) 
deve ser declarado pelo programador e não seguir o padrão da \acrshort{api}, que é utilizar as variáveis como \texttt{shared} 
\cite{unp:bosco-openmp-conceitos}. Quando declaradas como \texttt{shared}, as variáveis são compartilhadas entre 
as \textit{\gls{threads}} criadas pela \acrshort{openmp}. Já quando declaradas com a cláusula \texttt{private}, cada \textit{\gls{threads}} 
possui uma cópia da variável, sendo cada cópia isolada das demais \cite{unp:bosco-openmp-conceitos}.

Como a \acrshort{api} já sabe quantas iterações os laços \texttt{for} devem realizar (pela própria estrutura do laço), 
ela divide essa quantidade pelo número de \textit{\gls{threads}} determinadas pelo usuário, inicializando as variáveis 
de iteração \texttt{i} e \texttt{j} de acordo com a porção de iterações que cada \textit{\gls{threads}} recebeu para 
trabalhar.

\begin{itemize}
	\item explicar o código fake das diferenças finitas 1D.
\end{itemize}
    
    \section{Um contato mais profundo com as \textit{threads} - Pthreads}
Essa Seção visa apresentar os códigos ``falsos'' envolvendo a \acrshort{api} \acrfull{pthreads}.
O primeiro desses mostrará como dividir o trabalho da avaliação da função de um paraboloide 
entre \glspl{threads}. Já o segundo fará algo semelhante, mas considerando que o trabalho de todas as 
\glspl{threads} deve ter terminado para se iniciar a próxima iteração de cálculos. 

\subsection{Avaliação de uma função}
Para dividir o trabalho entre as \glspl{threads}, inseriu-se uma \gls{struct} no Código \ref{code:parabolloid-pthreads}, cuja as instâncias seriam dadas às \glspl{threads}, uma instância para cada. 
Nessa \gls{struct} existe um ponteiro para a matriz de cálculos, os limites desta determinados para cada \gls{threads}, os valores iniciais para esses limites e os valores de \textit{offset} para cada dimensão.

O funcionamento do Código se baseia na utilização da função \texttt{pthread\_create()} para lançar as \glspl{threads} que se utilizarão da função \texttt{calculate()} para executar os cálculos das suas respectivas porções da matriz. Existe também no Código a função \texttt{pthread\_join()}, que foi utilizada apenas para ilustração e, principalmente, treino para o assunto da próxima Seção.

\subsection{Propagação de onda em uma dimensão}
Possuindo uma estrutura semelhante ao código tratado pela Seção anterior, o Código \ref{code:fdm-1d-pthreads}, responsável por simular a propagação de uma onda unidimensional, possui como diferença essencial a necessidade de certificar que nenhuma outra \gls{threads} será criada antes que as atuais tenham terminado seu trabalho. 

Essa característica se deve ao fato de que não se pode calcular o próximo instante de tempo sem o anterior, visto como é definido o Método de Diferenças Finitas para a equação da onda. Dessa forma, o funcionamento da \texttt{phtread\_join()} se torna mais necessário com esse Código.
        
    \section{Realizando a Mescla de Pthreads e MPI}
Nessa Seção será apresentado o código ``falso'' que utilizou a \acrshort{api} \acrshort{mpi}, bem como os códigos ``falsos'' que utilizam dessa mesma \acrshort{api} como a \acrshort{pthreads}, mescladas. Por fim, é apresentado o código final do projeto, também com as mesmas \acrshort{api}s mescladas.

\subsection{Avaliação de uma Função com a MPI}
A lógica de programação com a \acrshort{mpi} para esse projeto consiste em envolver o código serial com as funções \texttt{MPI\_Init()} e \texttt{MPI\_Finalize()} e enviar os dados gerados pelas tarefas-servas para a tarefa-mestre, que deve salvar no disco o que foi recebido. 

É isso que ocorre no Código \ref{code:parabolloid-mpi}, com as tarefas-servas (\texttt{task\_id}$\neq 0$) enviando, com a função \texttt{MPI\_Send()}, seus dados para a tarefa-mestre (\texttt{task\_id}$= 0$). Esta recebe com \texttt{MPI\_Recv()}. 

\subsection{Propagação de Onda Unidimensional Utilizando MPI e Pthreads}
O Código \ref{code:fdm-1d-parallel} implementa a mesma simulação feita pelo Código \ref{code:fdm-1d-serial} de forma distribuída e dividida entre os processadores de uma máquina. 

Nele pode-se perceber trechos de código como os encontrados em \ref{code:fdm-1d-pthreads}, como a criação de \glspl{threads}, a espera pela finalização de sua execução, as instâncias de \glspl{struct} distribuídos a cada uma e uma função à parte utilizada pelas \glspl{threads} para realizarem os cálculos.

Da mesma forma, existem porções de código semelhantes às encontradas em \ref{code:parabolloid-mpi}, como a chamada das funções \texttt{MPI\_Init()} e \texttt{MPI\_Finalize()} e o envio de dados das tarefas-servas para a tarefa-mestre.

\subsection{O Código Final}
\todo[inline]{Colocar imagens dos traços}
Visto que até esse momento foram explicados exemplos de códigos paralelos construídos justamente para que o código objetivo (\ref{code:fdm-2d-parallel}) fosse concebido facil e didaticamente, a explicação da obtenção deste último será sucinta para evitar prolixidade.

Primeiramente, tendo-se o Código \ref{code:main-serial} como base, transferiu-se o trecho de cálculos para a simulação da propagação de onda bidimensional para uma função (\texttt{eval()}) e criou-se uma \gls{struct} que seria passada como argumento para as \glspl{threads} que executariam a função.

Por fora do laço em que as \glspl{threads} são criadas e terminadas existe o ambiente \acrshort{mpi}, possibilitando a execução distribuída e comunicação das tarefas. Essa comunicação ocorre próximo ao fim de \ref{code:fdm-2d-parallel}, com o uso das funções \texttt{MPI\_Send()} e \texttt{MPI\_Recv()}.
	
É importante notar que o código serial original exportava matrizes que poderiam ser usadas para construir imagens de contorno da onda em instantes de tempo arbitrários e \textit{arrays} que representam os traços. O código final, diferentemente, exporta apenas os \textit{arrays} dos traços. Isso foi feito para que o código obtido fosse mais profissional, visto que no ramo da sísmica se obtém apenas os traços \cite{notasAulas2018}, e para economizar espaço no disco do usuário.

    \chapter{Comparando o paralelo com o serial}

	
    
    \input{chapters/chp7/chp7.tex}

    % Printando a bibliografia
    \printbibliography

\end{document}
