              
%%%%%%%%%%%%%%%%%%%%%%%%%%%%%%%%%%%%%%%%%
% University Assignment Title Page 
% LaTeX Template
% Version 1.0 (27/12/12)
%
% This template has been downloaded from:
% http://www.LaTeXTemplates.com
%
% Original author:
% WikiBooks (http://en.wikibooks.org/wiki/LaTeX/Title_Creation)
%
% License:
% CC BY-NC-SA 3.0 (http://creativecommons.org/licenses/by-nc-sa/3.0/)
% 
% Instructions for using this template:
% This title page is capable of being compiled as is. This is not useful for 
% including it in another document. To do this, you have two options: 
%
% 1) Copy/paste everything between \begin{document} and \end{document} 
% starting at \begin{titlepage} and paste this into another LaTeX file where you 
% want your title page.
% OR
% 2) Remove everything outside the \begin{titlepage} and \end{titlepage} and 
% move this file to the same directory as the LaTeX file you wish to add it to. 
% Then add \input{./title_page_1.tex} to your LaTeX file where you want your
% title page.
%
%%%%%%%%%%%%%%%%%%%%%%%%%%%%%%%%%%%%%%%%%
\title{Uma introdução a Computação de Alto Desempenho}
%----------------------------------------------------------------------------------------
%   PACKAGES AND OTHER DOCUMENT CONFIGURATIONS
%----------------------------------------------------------------------------------------

\documentclass[12pt, a4paper]{report}
% Importando pacotes básicos
% Codificação
\usepackage[utf8]{inputenc}
% Linguagem
\usepackage[brazil]{babel}
% Fonte
\usepackage[T1]{fontenc}
% Determinando as margens do documento
\usepackage[left=3cm, right=2cm, bottom=2cm, top=3cm]{geometry}
% Possibilita o uso de LINKs e URLs ao longo do documento
\usepackage[pdftex, hidelinks]{hyperref}
% possibilita o uso do simbolo de graus 
\usepackage{gensymb}

% Pacotes que não lembro a funcionalidade
\usepackage{setspace}
\usepackage{graphicx} % Uso de figuras (ACHO)
\usepackage{float}
\usepackage{mathptmx}
\usepackage{amsmath}

% Permitir quebra de página nos ambientes de equações
\allowdisplaybreaks

% Acho que usei isso para colocar figuras em tabelas
\usepackage{subfig}

% Anexar PDFs no LaTeX
\usepackage[final]{pdfpages}

% Para colocar apêndices no LaTeX
\usepackage[toc,page]{appendix}

% Insercao de codigos no LaTeX
\usepackage{listings}
\lstset{language=Python}    % Setando a linguagem padrao

\usepackage{color}

\definecolor{codegreen}{rgb}{0,0.6,0}
\definecolor{codegray}{rgb}{0.5,0.5,0.5}
\definecolor{codepurple}{rgb}{0.58,0,0.82}
\definecolor{backcolour}{rgb}{0.95,0.95,0.92}

% Definindo estilo de exibição de código
\usepackage{color}
\lstdefinestyle{mystyle}{
	backgroundcolor=\color{backcolour},   
	commentstyle=\color{codegreen},
	keywordstyle=\color{magenta},
	numberstyle=\tiny\color{codegray},
	stringstyle=\color{codepurple},
	basicstyle=\footnotesize,
	breakatwhitespace=false,         
	breaklines=true,                 
	captionpos=b,                    
	keepspaces=true,                 
	numbers=left,                    
	numbersep=5pt,                  
	showspaces=false,                
	showstringspaces=false,
	showtabs=false,                  
	tabsize=4,
	basicstyle=\scriptsize
}

% Para adicionar notas e TODO's
\usepackage{todonotes}

% Para colocar margem da na legenda de um figure/table
% \usepackage[margincaption,outercaption,ragged,wide]{sidecap}
% \sidecaptionvpos{figure}{t} 
\usepackage[figurename=Fig.]{caption}
\DeclareCaptionStyle{figstyle}
  [format=plain,margin=0pt,justification=centering]
  {format=hang,calcmargin={0pt,\widthof{\captionfont\captionlabelfont\figurename~\thefigure: }},
   font=small,labelfont=bf}
\captionsetup[figure]{style=figstyle}

% Indenta o primeiro parágrafo das seções
\usepackage{indentfirst}

% Translating mathematical function names to Portuguese
\let\sin\relax        \DeclareMathOperator{\sin}{sen}
\let\arcsin\relax     \DeclareMathOperator{\arcsin}{arcsen}

\begin{document}

	\begin{titlepage}
	
	\newcommand{\HRule}{\rule{\linewidth}{0.5mm}} % Defines a new command for the horizontal lines, change thickness here
	
	\center % Center everything on the page
	 
	%----------------------------------------------------------------------------------------
	%   HEADING SECTIONS
	%----------------------------------------------------------------------------------------
	
	\textsc{\LARGE Centro Federal de Educação Tecnológica de Minas Gerais}\\[3.5cm] % Name of your university/college
	\textsc{\Large Engenharia de Computação}\\[3.5cm] % Major heading such as course name
	
	%----------------------------------------------------------------------------------------
	%   TITLE SECTION
	%----------------------------------------------------------------------------------------
	
	\HRule \\[0.4cm]
	{ \huge \bfseries Uma introdução a Computação de Alto Desempenho}\\[0.2cm] % Title of your document
	\HRule \\[2.5cm]
	 
	%----------------------------------------------------------------------------------------
	%   AUTHOR SECTION
	%----------------------------------------------------------------------------------------
	
	\begin{minipage}{0.4\textwidth}
	\begin{flushleft} \large
	\emph{Orientando:}\\ Marcelo Lopes de Macedo \textsc{Ferreira Cândido} % Your name
	\end{flushleft}
	\end{minipage}
	~
	\begin{minipage}{0.4\textwidth}
	\begin{flushright} \large
	\emph{Orientador:} \\
	Prof. Dr. Luis Alberto \textsc{D'Afonseca} % Supervisor's Name
	\end{flushright}
	\end{minipage}\\[7cm]
	
	% If you don't want a supervisor, uncomment the two lines below and remove the section above
	%\Large \emph{Author:}\\
	%John \textsc{Smith}\\[3cm] % Your name
	
	%----------------------------------------------------------------------------------------
	%   DATE SECTION
	%----------------------------------------------------------------------------------------
	
	{\large BELO HORIZONTE\\\today}\\[1cm] % Date, change the \today to a set date if you want to be precise
	
	%----------------------------------------------------------------------------------------
	%   LOGO SECTION
	%----------------------------------------------------------------------------------------
	
	%\includegraphics{logo.png}\\[1cm] % Include a department/university logo - this will require the graphicx package
	 
	%----------------------------------------------------------------------------------------
	
	\vfill % Fill the rest of the page with whitespace
	
	\end{titlepage}

    \lstset{style=mystyle}

    \tableofcontents
    
    \chapter{Introdução}
    
    \chapter{A arquitetura e a organização computacional}
    
	    \section{A arquitetura de von Neumann}
		    
	    \section{O processador}
		    
	    \section{A memória principal}
		    
	    \section{A cache}
	    
%	    \section{Extra: Placas de vídeo}
		    
	\chapter{Por que a paralelização é necessária?}
	
		\section{Conseguir mais em menos tempo}
		
%		\section{A barreira da memória - \textit{Memory wall}}
	
		\section{A barreira do paralelismo a nível de instruções - \textit{ILP wall}}
		
		\section{A barreira no gasto de energia dos processadores - \textit{Power wall}}
	
    \chapter{O programa serial}
    
	    \section{O programa-exemplo}
	    
	    \section{Otimizações sobre programas seriais}
    
    \chapter{OpenMP}
    
    \chapter{Pthreads}
    
    \chapter{MPI}
    
    \chapter{Clusters - o que são e como utilizá-los}
    
    \chapter{Comparações entre as programações sequencial e paralela}
    
    \chapter{Considerações Finais}
    
    % Definindo o estilo de bibliografia
	\bibliographystyle{abbrv}
	
	% Definindo o arquivo .bib de bibliografia
	\bibliography{BIB}

\end{document}