\section{Primeiro contato do código com as \textit{threads} - OpenMP}

Essa seção pretende demonstrar os esforços realizados para se construir dois
códigos ``falsos'' que, como já dito, foram criados para facilitar a criação do código final.
Esses códigos são variações dos códigos ``falsos'' seriais, realizando as mesmas tarefas
(avaliação de função e cálculo da propragação de ondas em meio unidimensional), mas
com o auxílio da \gls{api} \acrshort{openmp}.

\subsection{Avaliação de uma função}

O Código \ref{code:parabolloid-openmp} utiliza a estrutura
\begin{lstlisting}
#pragma omp parallel for default(none) shared(A, x_b, y_b, x_ofst, y_ofst, x_points, y_points) private(i, j, x_i, y_i)
\end{lstlisting}
cuja versão mais simples foi descrita na Seção \ref{sec:openmp},  para aplicar o paralelismo no problema. 

A estrutura \texttt{default(none)} serve para explicitar que o tipo das variáveis (\texttt{shared} ou \texttt{private}) 
deve ser declarado pelo programador e não seguir o padrão da \acrshort{api}, que é utilizar as variáveis como \texttt{shared} 
\cite{unp:bosco-openmp-conceitos}. Quando declaradas como \texttt{shared}, as varíaveis são compartilhadas entre 
as \textit{\gls{threads}} criadas pela \acrshort{openmp}. Já quando declaradas com a cláusula \texttt{private}, cada \textit{\gls{threads}} 
possui uma cópia da variável, sendo cada cópia isolada das demais \cite{unp:bosco-openmp-conceitos}.

Como a \acrshort{api} já sabe quantas iterações os laços \texttt{for} devem realizar (pela própria estrutura do laço), 
ela divide essa quantidade pelo número de \textit{\gls{threads}} determinadas pelo usuário, inicializando as variáveis 
de iteração \texttt{i} e \texttt{j} de acordo com a porção de iterações que cada \textit{\gls{threads}} recebeu para 
trabalhar.

\begin{itemize}
	\item explicar o código fake das diferenças finitas 1D.
\end{itemize}