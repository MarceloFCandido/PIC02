\section{A Construção do Código Serial}

\todo[inline]{Apresentar as imagens geradas pelos códigos seriais}

Como dito anteriormente na Seção \ref{subsec:parallel-design},
após se compreender o problema a ser programado em termos paralelos,
deve-se então criar o código serial que resolve o problema.

Para tal, criaram-se as \glspl{class} \texttt{\_2Dwave} (Código \ref{code:wave-serial}), que visa representar as características da
onda e também da sua fonte; \texttt{interface} (Código \ref{code:interface-serial}) e \texttt{velocity}
(Código \ref{code:velocidade-serial}), que buscam representar as características do meio em que a onda propagará.
Para unir essas informações e realizar os cálculos da propagação da onda utilizando o Método de Diferenças Finitas, foi
criado o programa principal (Código \ref{code:main-serial}).

\subsection{Camadas}
É importante relembrar que quando se fala de camadas se refere às porções de solo subterrâneas
que estão isoladas umas das outras por características diferentes e bem definidas (como o material
de que são feitas e o quanto esse está comprimido, por exemplo). 

Duas características do meio podem ser
usadas para a simulação da propagação das ondas: ângulo que as interfaces (limite de uma camada para a outra,
sendo representadas por funções de primeiro nesse projeto) possuem e a velocidade (representada por uma funcão de
primeiro grau com duas variáveis) das camadas.

\subsubsection{Interfaces}
Tanto no código serial quanto no final (paralelo), as interfaces das camadas são representadas
pela \gls{class} \texttt{interface}. Tal \gls{class} possui dois atributos: \texttt{a} (coeficiente angular) e
\texttt{b} (termo constante). Além disso, ela também possui um método, \texttt{getY(\textbf{double} x)}, responsável por retornar
o $y$ da reta para um dado $x$.

\subsubsection{Velocidades}
Já quanto às velocidades das camadas, no código serial e no final (paralelo), elas são representadas
pela \gls{class} \texttt{velocity}. Essa \gls{class} possui três atributos: \texttt{a}, \texttt{b} e \texttt{c},
sendo os dois primeiros os multiplicadores das variáveis $x$ e $y$ da função que representa a velocidade; e o último
sendo o termo constante. A \gls{class} também possui o método \texttt{getGradientVelocity(\textbf{double} x, \textbf{double} y)}
que retorna a velocidade encontrada em um determinado ponto $(x, y)$ da camada.

\subsection{Onda}
A onda bidimensional que será propagada através do meio é representada pela \gls{class} \texttt{\_2Dwave}, que possui como
atributos: a extensão do domínio em $x$ (\texttt{Lx}) e $y$ (\texttt{Ly}), o tempo máximo que a simulação
da propagação deve durar (\texttt{tMax}), número de pontos em $x$ (\texttt{Mx}) e $y$ (\texttt{Ny}), para
a discretização do domínio; a frequência da onda (\texttt{w}), sua amplitude (\texttt{A}), seu ponto $(x, y)$ e tempo iniciais
(\texttt{Xp}, \texttt{Yp} e \texttt{Tp}).

Além desses atributos, a \gls{class} também possui os métodos
\texttt{evaluateFXYT(\textbf{double} x, \textbf{double} y, \textbf{t})}, que retorna a amplitude da onda para determinado ponto
$(x, y, t)$; e \newline\texttt{getVelocitiesMatrix(\textbf{vector<interface>} interfaces, \textbf{vector<velocity>} velocities)},
que retorna uma matriz da velocidade de cada ponto $(x, y)$ do domínio.

\subsection{O Programa Principal}
O programa principal consiste, basicamente, em recolher as informações dadas pelo usuário por meio do Código
\ref{code:cli-serial}; em seguida, realizar os cálculos do Método de Diferenças Finitas e, por fim, salvar
os dados resultantes desses cálculos. Esses dados podem ser matrizes, das quais se pode obter imagens da onda propagando; ou vetores, dos
quais se pode obter os traços sísmicos registrados pelos receptores simulados.

\subsection{Os Códigos Auxiliares ``Falsos''}
Também foram criados códigos seriais ``falsos'' para que a construção do código final acontecesse de forma mais gradual e didática.
São chamados ``falsos'' porque buscam resolver problemas mais fáceis que o problema real que o projeto busca solucionar. Sobre
eles foram construídos níveis de paralelização como forma de estudo para que os mesmos níveis a serem implementadas no código real
fossem alcançados mais facilmente.

O primeiro desses códigos (Código \ref{code:parabolloid-serial}) preenche uma matriz com os valores em $z$ do paraboloide $z = 2x^2 + y^2$,
sendo os limites de $x$ e $y$ delimitados pelo usuário. Já o segundo (Código \ref{code:fdm-1d-serial}) calcula a propagação de uma onda
unidimensional sobre uma corda ao longo do tempo, cujos limites também são definidos pelo usuário.

O primeiro código citado propõe a introdução facilitada de níveis de paralelização, não havendo nenhum problema que exija
um nível maior de atenção. Já o segundo, por conta de que os cálculos do próximo instante de tempo $t_1$ dependem de que os cálculos de $t_0$
(com $t_0 < t_1$) estejam finalizados, insere o problema da sincronização ao se aplicar paralelismo com \gls{threads}. Estudar esse problema
é essencial antes de se implementar o código final.

