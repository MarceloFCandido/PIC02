
\chapter{A Arquitetura Computacional Atual e a Necessidade de Paralelização}

Para compreender a questão da paralelização envolvida nesse trabalho é
necessário entender de onde e porque ela veio. Para tal, é necessário
se apresentar as principais peças que constituem um computador moderno,
os problemas que surgiram no processo de evolução da \textbf{arquitetura
computacional} convencional (baseando-se na \gls{von-Neumann}) e como isso culminou no 
paralelismo \cite{LLNL:parcomp}.

Antes de iniciar esse processo, é também necessário se explicar o que se
entende por arquitetura computacional. Trata-se da área do conhecimento
que estuda a interface entre \textit{software} e \textit{hardware}, desde
o mais baixo nível, no qual o processador manipula as informações
(\gls{instr-machine} e dados) entregues a ele, para toda operação realizada no
computador. Após esse nível, tem-se as políticas de manipulação de dados nas memórias
cache (e seus níveis), de acesso aleatório (\acrshort{RAM}) e de 
armazenamento não-volátil (discos rígidos, por exemplo).
Por fim, chega-se à interação dos computadores com os demais periféricos que
por ventura estão nele conectados, realizando \textit{inputs} (entradas)
e/ou \textit{outputs} (saídas), também conhecidas pela abreviação "I/O",
como teclado, \textit{mouse}, monitor, etc \cite{Catsoulis}.

O presente capítulo abordará alguns componentes da arquitetura computacional 
(processador e memória \acrshort{RAM}) de uma forma básica, além de explicar por quais 
motivos a paralelização se tornou necessária e como ela se desenvolveu.

\section{O Processador}

\label{sec:processor}

Um processador consiste em um módulo de \textit{hardware} capaz de manipular 
\gls{instr-machine} armazenadas
em memória e produzir os resultados desejados através dessas instruções. Tais resultados
podem ser de cunho ou lógico-aritmético ou manipulação de dados, no geral. Tal módulo é
indispensável para o conceito de computadores como conhecemos hoje, de tal forma que, se
não fosse pela necessidade de memória para a armazenagem de dados, um processador poderia
ser a definição de um computador.

\section{A memória principal}

    \begin{itemize}
        \item introdução do que é uma memória;
        \item apresentação básica dos tipos de memória 
        (enfase na RAM);
        \item apresentação básica das propriedades de uma 
        memória RAM.
    \end{itemize}

\section{Conseguir mais em menos tempo}

	A computação sempre foi uma das principais ferramentas humanas para avanços 
	tecnológicos nos últimos séculos. Seja na engenharia, medicina, área 
	militar, biologia, química, sísmica e até no cinema, os computadores tem 
	sido utilizados em tarefas como mecânicas, estudos patológicos, 
	ataques/defesas nacionais, enovelamento de proteínas, dinâmica molecular, 
	monitoramento sísmico e animações tridimensionais.
	
	Todas essas áreas costumam requerer que os resultados obtidos 
	computacionalmente sejam gerados o mais rápido possível. Além disso, 
	muitas (senão quase todas) as \textbf{simulações científicas}\footnote{
		explicar o que seriam simulações científicas}
	envolvidas não podem ser executadas em tempo hábil com o tipo de processador
	visto na Seção \ref{sec:processor}. Visto 
	isso, os projetistas por trás dos projetos de processadores precisaram 
	implementar mecanismos nos modelos para explorá-los ao máximo. Nessa tarefa, 
	encontraram barreiras \todo{Ir à biblioteca}\cite{autor:intro-par-comp}.
    
    \subsection{A barreira do paralelismo a nível de instruções - \textit{ILP wall}}
    
    	Vimos na Seção \ref{sec:processor} um processador com a capacidade de 
    	executar uma instrução por ciclo de \textit{clock}. Com isso, a duração 
    	do ciclo devia ser a mesma da execução da instrução mais demorada. 
    	Na necessidade de se adquirir mais velocidade de processamento, 
    	os projetistas perceberam que podiam particionar as instruções, de modo 
    	a executar cada estágio resultante em um ciclo de \textit{clock}, 
    	deixando o resultado para a próxima partição a ser operada. Assim 
    	nasceu o processador \textbf{multiciclo}. Dessa forma, a duração 
    	do ciclo de \textit{clock} reduziu para a mesma do estágio mais 
    	demorado dentre as instruções.
    	
    	Contudo, a necessidade de se acelerar os processadores continuava. Em
    	seguida, os projetistas perceberam que as unidades funcionais dos 
    	processadores ficavam ociosas quando não se tratava do estágio de uma 
    	instrução em que elas eram utilizadas. Era então possível executar uma
    	instrução ao mesmo tempo que outra, desde que ambas se encontrassem, 
    	cada uma, em estágios $A$ e $B$, sendo $A$ o estágio da instrução mais 
    	antiga e $B$ o da mais nova,  com $B = A - 1$. 
    	
    	Ou seja, por exemplo, 
    	considere uma instrução $i$ liberada no tempo de \textit{clock} $1$, 
    	passando por seu primeiro estágio. No tempo $2$, ela estará em sua
    	segunda etapa e as unidades funcionais responsáveis pela primeira 
    	estariam ociosas, se não fosse pela inovação apresentada acima. Com 
    	esta, uma nova instrução $j$ é buscada ainda nesse tempo, tendo 
    	seu primeiro estágio executado. No tempo de ciclo de \textit{clock} 3,
    	a instrução $i$ passará para o seu terceiro estágio, enquanto a $j$
    	passará para o segundo e uma nova instrução poderá ser buscada. À essa 
    	inovação, foi dado o nome de \textbf{paralelismo a nível de instruções}, 
    	visto que se tem mais de uma instrução sendo executada ao mesmo tempo e 
    	uma pronta a cada ciclo de \textit{clock}.
    	
    	Em seguida, os projetistas
    
        \begin{itemize}
            \item Explicar os conceitos de superpipeline e superescalar;
            \item Explicar que, ao se estender muito um pipeline, temos problemas;
            \item Explicar os problemas da superescalaridade
        \end{itemize}
    
    \subsection{A barreira no gasto de energia dos processadores - \textit{Power wall}}
    
        \begin{itemize}
            \item Introduzir a lei de Moore;
            \item Explicar que houve evolução na taxa de clock ao longo do tempo;
            \item Explicar que essa evolução foi amortecida nos últimos anos devido 
            ao gasto energético e às altas temperaturas que os processadores alcançaram 
        \end{itemize}
    
    \subsection{A barreira da memória - \textit{Memory wall}}
    
	    \todo[inline]{Conferir se existe mesmo a memory wall}
	    \todo[inline]{Revisar o que é a memory wall e completar o itemize abaixo}
	    
	    \begin{itemize}
	    	\item 
	    \end{itemize}

\section{Paralelismo - a alternativa para se contornar as barreiras}

	\label{sec:parallelism}
    
    Visto a incapacidade de se transpor o alto gasto energético das 
    unidades de processamento, junto com as consequentes altas temperaturas,
    além da incapacidade de se aumentar o número de instruções prontas por 
    ciclo de \textit{clock}, precisamos de uma alternativa para se continuar 
    aumentando o poder de processamento dos computadores. 
    
    Tal alternativa foi então aumentar o número de unidades de processamento,
    seja em um \textit{chip} com mais de uma unidade (chamada núcleo, ou 
    \textbf{\textit{core}} em inglês) ou em um sistema com mais de uma 
    máquina, que por sua vez possui uma ou mais unidades de processamento. 
    Dessa forma, o número de tarefas concluídas por unidade de tempo aumentou,
    visto.
    A isso foi dado o nome de \textbf{paralelismo}.
    
    Existem muitos motivos para se utilizar o paralelismo. O mais importante é
    que o universo possui muitos processos paralelos, ou seja, ocorrendo ao 
    mesmo tempo, tais como mudanças climáticas, montagens de veículos e 
    aeronaves, tráfico em um pedágio e acessos a um site. Em consequência da 
    necessidade de se processar esses eventos por computadores, surgem outros motivos para a computação paralela:
    \begin{itemize}
    	\item poupar tempo e/ou dinheiro: existem problemas computacionais 
    	que são muito demorados, senão impossíveis, para se resolver
    	serialmente. Dessa forma, usando-se a computação serial, se paga mais,
    	visto o uso por mais tempo do poder computacional;
    	\item resolver problemas maiores e/ou mais complexos, visto o
    	processamento mais rápido e a possibilidade de se dividir o trabalho
    	mais máquinas;
    	\item prover concorrência: realizar várias tarefas simultaneamente
    	\cite{LLNL:parcomp}.
    \end{itemize}
    
    Quando se fala em paralelismo, é importante mostrar que existem diferentes
    disposições de memória nesse meio, o que veremos na subseção
    \ref{subsec:memory-architectures}. Além disso, existem diferentes
    paradigmas de paralelismo, sendo alguns brevemente explicados na subseção
    \ref{subsec:par-comp-models}. Há também outros assuntos a serem tratados 
    ao se projetar programas paralelos, sendo alguns abordados na subseção 
    \ref{subsec:parallel-design}. Fora isso, mais informações sobre 
    paralelismo podem ser encontradas em Barney \cite{LLNL:parcomp}.
    
    \subsection{Arquiteturas de memória na computação paralela}
    
	    \label{subsec:memory-architectures}
	    
	    Na computação paralela, existem diferentes formas pelas quais a 
	    memória é administrada e interpretada. Isso pode ser realizado fisica 
	    e/ou logicamente, principalmente de três formas:
        \begin{itemize}
            \item memória compartilhada: cada unidade de processamento de
            um sistema tem acesso à toda a memória, com um \textbf{espaço de
            endereçamento}\footnote{\textbf{espaço de endereçamento}, segundo
            a Wikipédia \cite{wiki:address}, é uma ``série de endereços
            discretos'' que, no caso, nomeiam as posições de memória} global;
            
            \item memória distribuída: cada unidade de processamento possui
            sua própria memória, mapeada apenas para ele e que pode ser 
            acessada pelos demais nós através de requisições feitas por meio
            de uma rede que os liga. Além disso, cada nó opera
            independentemente, visto que cada um tem sua própria memória.
            Dessa forma, as mudanças que um nó opera sobre sua própria 
            memória não afetam as dos demais nós;
            
            \item híbrida: literalmente a mescla de ambas, ou seja, cada 
            \textbf{nó}\footnote{\textbf{nó} é o nome dado na computação 
            paralela para cada unidade de processamento (ou conjunto dessas) 
            independente.} possui mais de uma unidade de processamento e sua 
	        própria memória, sendo também capaz de acessar as dos outros 
	        \cite{LLNL:parcomp}.
        \end{itemize}
        
    
    \subsection{Modelos da computação paralela}
    
		\label{subsec:par-comp-models}
		Os modelos de programação paralela 
    
        \begin{itemize}
            \item Shared Memory Model;
            \item Threads Model;
            \item Distributed Memory / Message Passing Model;
            \item Data Parallel Model;
            \item Hybrid Model
        \end{itemize}
    
    \subsection{Desenhando programas paralelos}
    
	    \label{subsec:parallel-design}
    
        \begin{itemize}
            \item explicar a diferença entre a paralelização automática e a 
            manual;
            \item explicar a necessidade de se entender o problema e o programa;
            \item explicar
            \begin{itemize}
            	\item particionamento;
            	\item sincronização.
            \end{itemize}

        \end{itemize}