\section{Conseguir mais em menos tempo}

	A computação sempre foi uma das principais ferramentas humanas para avanços 
	tecnológicos nos últimos séculos. Seja na engenharia, medicina, área 
	militar, biologia, química, sísmica e até no cinema, os computadores tem 
	sido utilizados em tarefas como mecânicas, estudos patológicos, 
	ataques/defesas nacionais, enovelamento de proteínas, dinâmica molecular, 
	monitoramento sísmico e animações tridimensionais.
	
	Todas essas áreas costumam requerer que os resultados obtidos 
	computacionalmente sejam gerados o mais rápido possível. Além disso, 
	muitas (senão quase todas) as \textbf{simulações científicas}\footnote{
		explicar o que seriam simulações científicas}
	envolvidas não podem ser executadas em tempo hábil com o tipo de processador
	visto na Seção \ref{sec:processor}. Visto 
	isso, os projetistas por trás dos projetos de processadores precisaram 
	implementar mecanismos nos modelos para explorá-los ao máximo. Nessa tarefa, 
	encontraram barreiras \todo{Ir à biblioteca}\cite{autor:intro-par-comp}.
    
    \subsection{A barreira do paralelismo a nível de instruções - \textit{ILP wall}}
    
    	Vimos na Seção \ref{sec:processor} um processador com a capacidade de 
    	executar uma instrução por ciclo de \textit{clock}. Com isso, a duração 
    	do ciclo devia ser a mesma da execução da instrução mais demorada. 
    	Na necessidade de se adquirir mais velocidade de processamento, 
    	os projetistas perceberam que podiam particionar as instruções, de modo 
    	a executar cada estágio resultante em um ciclo de \textit{clock}, 
    	deixando o resultado para a próxima partição a ser operada. Assim 
    	nasceu o processador \textbf{multiciclo}. Dessa forma, a duração 
    	do ciclo de \textit{clock} reduziu para a mesma do estágio mais 
    	demorado dentre as instruções.
    	
    	Contudo, a necessidade de se acelerar os processadores continuava. Em
    	seguida, os projetistas perceberam que as unidades funcionais dos 
    	processadores ficavam ociosas quando não se tratava do estágio de uma 
    	instrução em que elas eram utilizadas. Era então possível executar uma
    	instrução ao mesmo tempo que outra, desde que ambas se encontrassem, 
    	cada uma, em estágios $A$ e $B$, sendo $A$ o estágio da instrução mais 
    	antiga e $B$ o da mais nova,  com $B = A - 1$. 
    	
    	Ou seja, por exemplo, 
    	considere uma instrução $i$ liberada no tempo de \textit{clock} $1$, 
    	passando por seu primeiro estágio. No tempo $2$, ela estará em sua
    	segunda etapa e as unidades funcionais responsáveis pela primeira 
    	estariam ociosas, se não fosse pela inovação apresentada acima. Com 
    	esta, uma nova instrução $j$ é buscada ainda nesse tempo, tendo 
    	seu primeiro estágio executado. No tempo de ciclo de \textit{clock} 3,
    	a instrução $i$ passará para o seu terceiro estágio, enquanto a $j$
    	passará para o segundo e uma nova instrução poderá ser buscada. Ao 
    	``trem'' de estágios foi dado o nome de \textbf{\textit{pipeline}}. À essa 
    	inovação, foi dado o nome de \textbf{paralelismo a nível de instruções}, 
    	visto que se tem mais de uma instrução sendo executada ao mesmo tempo e 
    	uma pronta a cada ciclo de \textit{clock}.
    	
    	Em seguida, os projetistas decidiram construir estágios menores, 
    	consequentemente aumentando a frequência de \textit{clock} (que é 
	    	$\frac{1}{\text{tempo de ciclo do \textit{clock}}}$ 
    	e dada em hertz [Hz]), o tamanho do \textit{pipeline} e o número de 
    	instruções operadas simultaneamente. Nasceu o \textbf{superpipeline}. 
    	
    	Apesar disso aumentar a performance dos processadores, ao se 
    	incrementar o número de estágios para um valor maior que \todo{buscar fonte e confirmação} oito tem-se
    	o desempenho degradado.
    	
    	Por fim, os projetistas de \textbf{hardware} 
    
        \begin{itemize}
            \item Explicar os conceitos de superpipeline e superescalar;
            \item Explicar os problemas da superescalaridade
        \end{itemize}
    
    \subsection{A barreira no gasto de energia dos processadores - \textit{Power wall}}
    
        \begin{itemize}
            \item Introduzir a lei de Moore;
            \item Explicar que houve evolução na taxa de clock ao longo do tempo;
            \item Explicar que essa evolução foi amortecida nos últimos anos devido 
            ao gasto energético e às altas temperaturas que os processadores alcançaram 
        \end{itemize}
    
    \subsection{A barreira da memória - \textit{Memory wall}}
    
	    \todo[inline]{Conferir se existe mesmo a memory wall}
	    \todo[inline]{Revisar o que é a memory wall e completar o itemize abaixo}
	    
	    \begin{itemize}
	    	\item 
	    \end{itemize}