\section{Conseguir mais em menos tempo}

	A computação sempre foi uma das principais ferramentas humanas para avanços 
	tecnológicos nos últimos séculos. Seja na engenharia, medicina, área 
	militar, biologia, química, sísmica e até no cinema, os computadores tem 
	sido utilizados em tarefas como mecânicas, estudos patológicos, 
	ataques/defesas nacionais, enovelamento de proteínas, dinâmica molecular, 
	monitoramento sísmico e animações tridimensionais.
	
	Todas essas áreas costumam requerer que os resultados obtidos 
	computacionalmente sejam gerados o mais rápido possível. Além disso, 
	muitas (senão quase todas) as \textbf{simulações científicas}\footnote{
		explicar o que seriam simulações científicas}
	envolvidas não podem ser executadas em tempo hábil com o tipo de processador
	visto na seção \todo{Consertar essa referência}\ref{sec:processor}. Visto 
	isso, os \textit{designers} por trás dos projetos de processadores precisaram 
	implementar mecanismos nos modelos para explorá-los ao máximo. Nessa tarefa, 
	encontraram barreiras \todo{text}\cite{}.
    
    \subsection{A barreira do paralelismo a nível de instruções - \textit{ILP wall}}
    
    	
    
        \begin{itemize}
            \item Explicar do que se trata o paralelismo a nível de instruções;
            \item Explicar os conceitos de superpipeline e superescalar;
            \item Explicar que, ao se estender muito um pipeline, temos problemas;
            \item Explicar os problemas da superescalaridade
        \end{itemize}
    
    \subsection{A barreira no gasto de energia dos processadores - \textit{Power wall}}
    
        \begin{itemize}
            \item Introduzir a lei de Moore;
            \item Explicar que houve evolução na taxa de clock ao longo do tempo;
            \item Explicar que essa evolução foi amortecida nos últimos anos devido 
            ao gasto energético e às altas temperaturas que os processadores alcançaram 
        \end{itemize}
    
    \subsection{A barreira da memória - \textit{Memory wall}}
    
	    \todo[inline]{Conferir se existe mesmo a memory wall}
	    \todo[inline]{Revisar o que é a memory wall e completar o itemize abaixo}
	    
	    \begin{itemize}
	    	\item 
	    \end{itemize}