\section{Conseguir mais em menos tempo}

    \begin{itemize}
        \item A necessidade de se aumentar a velocidade de processamento;
        \item introdução aos problemas encontrados na computação serial
    \end{itemize}

    \subsection{A barreira da memória - \textit{Memory wall}}
    
        \todo[inline]{Conferir se existe mesmo a memory wall}
        \todo[inline]{Revisar o que é a memory wall e completar o itemize abaixo}
    
        \begin{itemize}
            \item 
        \end{itemize}
    
    \subsection{A barreira do paralelismo a nível de instruções - \textit{ILP wall}}
    
        \begin{itemize}
            \item Explicar do que se trata o paralelismo a nível de instruções;
            \item Explicar os conceitos de superpipeline e superescalar;
            \item Explicar que, ao se estender muito um pipeline, temos problemas;
            \item Explicar os problemas da superescalaridade
        \end{itemize}
    
    \subsection{A barreira no gasto de energia dos processadores - \textit{Power wall}}
    
        \begin{itemize}
            \item Introduzir a lei de Moore;
            \item Explicar que houve evolução na taxa de clock ao longo do tempo;
            \item Explicar que essa evolução foi amortecida nos últimos anos devido 
            ao gasto energético e às altas temperaturas que os processadores alcançaram 
        \end{itemize}