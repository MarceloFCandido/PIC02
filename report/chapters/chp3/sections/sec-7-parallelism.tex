\section{Paralelismo - a alternativa para se contornar as barreiras}

    \begin{itemize}
        \item em seguida, explicar porquê ele é uma saída possível
    \end{itemize}
    
    Visto a incapacidade de se transpor o alto gasto energético das 
    unidades de processamento, junto com as consequentes altas temperaturas,
    além da incapacidade de se aumentar o número de instruções prontas por 
    ciclo de \textit{clock}, precisamos de uma alternativa para se continuar 
    aumentando o poder de processamento dos computadores. 
    
    Tal alternativa foi então aumentar o número de unidades de processamento,
    seja em um \textit{chip} com mais de uma unidade (chamada núcleo, ou 
    \textbf{\textit{core}} em inglês) ou em um sistema com mais de uma 
    máquina, que por sua vez possui uma ou mais unidades de processamento. 
    Dessa forma, o número de tarefas concluídas por unidade de tempo aumentou,
    visto.
    A isso foi dado o nome de \textbf{paralelismo}.
    
    Existem muitos motivos para se utilizar o paralelismo. O mais importante é
    que o universo possui muitos processos paralelos, ou seja, ocorrendo ao 
    mesmo tempo, tais como mudanças climáticas, montagens de veículos e 
    aeronaves, tráfico em um pedágio e acessos a um site. Em consequência da 
    necessidade de se processar esses eventos por computadores, surgem outros motivos para a computação paralela:
    \begin{itemize}
    	\item poupar tempo e/ou dinheiro: existem problemas computacionais 
    	que são muito demorados, senão impossíveis, para se resolver
    	serialmente. Dessa forma, usando-se a computação serial, se paga mais,
    	visto o uso por mais tempo do poder computacional;
    	\item resolver problemas maiores e/ou mais complexos, visto o
    	processamento mais rápido e a possibilidade de se dividir o trabalho
    	mais máquinas;
    	\item prover concorrência: realizar várias tarefas simultaneamente,
    	 \cite{LLNL:parcomp}.
    \end{itemize}
    
    \subsection{Arquiteturas de memória na computação paralela}
    
        \begin{itemize}
            \item Explicar as arquiteturas:
            \begin{itemize}
                \item de memória compartilhada;
                \item memória distribuída;
                \item híbrida
            \end{itemize}
        \end{itemize}
    
    \subsection{Modelos da computação paralela}
    
        \begin{itemize}
            \item Shared Memory Model;
            \item Threads Model;
            \item Distributed Memory / Message Passing Model;
            \item Data Parallel Model;
            \item Hybrid Model
        \end{itemize}
    
    \subsection{Desenhando programas paralelos}
    
        \begin{itemize}
            \item explicar a diferença entre a paralelização automática e a 
            manual;
            \item explicar a necessidade de se entender o problema e o programa;
            \item explicar
            \begin{itemize}
            	\item particionamento;
            	\item sincronização.
            \end{itemize}
        \end{itemize}