\section{O Que São Aquisições Sísmicas e Como Modelá-las}

	No ramo da mineração não se pode tentar a esmo a descoberta de recursos
	minerais no subterrâneo de um local em que se já se suspeita sua existência.
	É necessário que, de alguma forma, se obtenha a forma dessa estrutura oculta
	para se saber os pontos onde se encontram as jazidas/poços desse recurso.
	A obtenção dos dados de como é essa estrutura se chama \textbf{aquisição
	sísmica}.

	A forma de se realizar essa aquisição pode variar com o ambiente e os
	métodos de coleta de dados e de processamento dos mesmos adotado. Os
	recursos minerais desejados podem se encontrar tanto em meios terrestres
	e/ou subaquáticos. Contudo, o meio em que a aquisição será realizada pouco
	importa nesse trabalho, o que será melhor explicado mais a
	frente.

	Quanto aos métodos de processamento utilizados, vale lembrar \todo{colocar
	isso no Capítulo \ref{chp1}} que, nesse trabalho, simularemos um problema
	direto, ou seja, tendo o meio (nesse caso, não-homogêneo) da aquisição,
	veremos como as ondas se propagam nele. Para tal, usaremos um método
	matemático específico. Nesse trabalho, como já foi dito no Capítulo
	\ref{chp1}, é o de Diferenças Finitas.

	Já para a coleta dos dados a serem processados podemos citar dois exemplos
	de aquisição
	\begin{enumerate}
		\todo[inline]{buscar referências}
		\item \textbf{marítima}: um navio equipado com um canhão
		sonoro emite ondas sonoras cujas reflexões e refrações nas camadas
		terrestres submarinas são captadas por filas de hidrofones puxadas pelo
		mesmo navio.

		\item \textbf{terrestre}: um explosivo é colocado (preferencialmente
		enterrado) em um terreno. Sua explosão gera uma onda sonora cujas
		reflexões são captadas por geofones distribuídos relativamente próximos,
		na superfície.
	\end{enumerate}
	Costuma-se alterar a posição da fonte sonora na realização da aquisição para
	\todo{conferir se procede} facilitar o modelamento do meio de propagação das
	ondas.
