\section{O Método de Diferenças Finitas (MDF)}

	Uma equação diferencial parcial, que geralmente é considerada em um domínio 
	contínuo, pode ser discretizada. Isso é feito para que a equação possa ser 
	representada e resolvida computacionalmente. 
	
	No caso do Método de Diferenças	Finitas para esse trabalho, basta 
	transcrever cada termo da equação para o equivalente na fórmula de 
	diferenças finitas nas derivações de segundo grau. Podemos 
	ver um exemplo disso na Equação \ref{eq:stencil1D}, para o caso da parte 
	espacial em $x$ da equação. 
	\begin{equation}
	    \dfrac{\partial^2 u}{\partial x^2} \xrightarrow{} \dfrac{u_{i-1,j,k} - 
	    2u_{i,j,k} + u_{i+1,j,k}}{\Delta x^2}
	    \label{eq:stencil1D}
	\end{equation}
	
	Para entender o que 
	significa o índice $i$ visto na Equação acima pode-se seguir uma analogia: 
	suponhamos um \textit{array} com mais que três posições. A posição $i$ seria 
	qualquer posição intermediária, $i-1$ a antecessora e a $i+1$ sucessora. Tal 
	estrutura é chamada de \textbf{estêncil}. O mesmo vale para os índices
	$j$ e $k$, mas para um \textit{array} com três dimensões. Podemos ver 
	uma alegoria dessa analogia na Figura \ref{fig:stencil1D}.
%	\begin{figure}
%		\centering
%       \label{fig:stencil1}
%		\includegraphics[scale=.5, width=\textwidth]{imagefile}
%		\caption{\textit{Stencil} unidimensional}
%	\end{figure}
	
	No caso desse trabalho, para a simulação da propagação de ondas em um 
	meio bidimensional ao longo do tempo, teremos que usar três estênceis 
	(os dois restantes podem ser vistos nas Equações 
	\ref{eq:stencils3D-1} e \ref{eq:stencils3D-2}), dois para as dimensões 
	espaciais e um para a temporal. Para tal, 
	podemos utilizar outra analogia: um \textit{array} tridimensional, onde cada
	plano de posições no espaço é um instante no tempo.
	\begin{align}
	    \dfrac{\partial^2 u}{\partial y^2} &\xrightarrow{} \dfrac{u_{i,j-1,k} - 2u_{i,j,k} + 
	        u_{i,j+1,k}}{\Delta y^2} 
	    \label{eq:stencils3D-1}
	    \\
		\dfrac{\partial^2 u}{\partial t^2} &\xrightarrow{} \dfrac{u_{i,j,k-1} - 2u_{i,j,k} + 
			u_{i,j,k+1}}{\Delta t^2}
	    \label{eq:stencils3D-2}
	\end{align}
	
	%	\begin{figure}[H]
	%		\centering
	%       \label{fig:stencil3D}
	%		\includegraphics[scale=.5, width=\textwidth]{imagefile}
	%		\caption{Estêncil tridimensional}
	%	\end{figure}
	
	Contudo, até então, não falamos sobre o Método de Diferenças Finitas 
	em si. Trata-se de uma sequência de iterações que marcham em função de
	alguma variável. No caso desse trabalho, o avanço se dá no tempo. Essa marcha 
	no tempo quer dizer que o valor para um ponto no espaço no próximo instante de 
	tempo será calculado com base nos valores para pontos no espaço em instantes 
	anteriores. Vamos ser explícitos. Temos que a equação \ref{eq:waveEq} traduzida 
	nas fórmulas vistas nas Equações 
	\ref{eq:stencil1D}, \ref{eq:stencils3D-1} e \ref{eq:stencils3D-2} se dá por
	\begin{equation}
    	\begin{split}
    		 \dfrac{u_{i,j,k-1} - 2u_{i,j,k} + u_{i,j,k+1}}{\Delta t^2} = \dfrac{1}{v^2}(\dfrac{u_{i-1,j,k} - 2u_{i,j,k} + u_{i+1,j,k}}{\Delta x^2} \\+ \dfrac{u_{i,j-1,k} - 2u_{i,j,k} + u_{i,j+1,k}}{\Delta y^2}) + f(x, y, t)
    	\end{split}
	\end{equation}
	onde $u_{k+1}$ é o ponto com valor a ser calculado para o próximo
	instante. Logo, ele precisa ser isolado, o que é mostrado na equação 
	\ref{eq:timeMarching}
	\begin{equation}
    	\begin{split}
    		\label{eq:timeMarching}
    		u_{i, j,k+1} = \dfrac{\Delta t^2}{v^2} \left(\dfrac{u_{i-1,j,k} - 2u_{i,j,k} + u_{i+1,j,k}}{\Delta x^2} + \dfrac{u_{i,j-1k} - 2u_{i,j,k} + u_{i,j+1,k}}{\Delta y^2}\right) + \\\Delta t^2f(x, y, t) - u_{i,j,k-1} + 2u_{i,j,k}
    	\end{split}
    \end{equation}