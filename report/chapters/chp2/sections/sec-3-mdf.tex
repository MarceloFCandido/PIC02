\section{O Método de Diferenças Finitas (MDF)}

	Uma equação diferencial parcial, que geralmente é considerada em um domínio 
	contínuo, pode ser discretizada. Isso é feito para que a equação possa ser 
	representada e resolvida computacionalmente. 
	
	No caso do método de diferenças	finitas para esse trabalho, basta 
	transcrever cada termo da equação para o equivalente na fórmula de 
	diferenças finitas para derivações de segundo grau, sobre a qual podemos 
	ver um exemplo na Tabela \ref{tab:stencil1D} para o caso da parte espacial em $x$ da equação. 
	\begin{table}[H]
		\caption{Fórmula de Diferenças Finitas para cálculo de derivada de segunda ordem, aqui representando a derivada na dimensão espacial $x$}
		\centering
		\begin{tabular}{c|c}
			\hline
			$\dfrac{\partial^2 u}{\partial x^2}$ & $\dfrac{u_{i-1,j,k} - 2u_{i,j,k} + 
				u_{i+1,j,k}}{\Delta x^2}$ \\ \hline
		\end{tabular}
		\label{tab:stencil1D}
	\end{table}
	
	
	Para entender o que 
	significa o índice $i$ visto na tabela acima é bom se seguir uma analogia: 
	suponhamos um \textit{array} de tamanho maior que três. A posição $i$ seria 
	qualquer posição intermediária, $i-1$ a antecessora e a $i+1$ sucessora. Tal 
	estrutura é chamada de \textit{stencil}. O mesmo vale para os índices
	$j$ e $k$. Podemos ver uma alegoria dessa 
	analogia na Figura \ref{fig:stencil1}.
	
	\todo[inline]{Inserir uma imagem para o stencil 1D}
%	\begin{figure}
%		\centering
%       \label{fig:stencil1}
%		\includegraphics[scale=.5, width=\textwidth]{imagefile}
%		\caption{\textit{Stencil} unidimensional}
%	\end{figure}
	
	No caso desse trabalho, para a simulação da propagação de ondas em um 
	meio bidimensional ao longo do tempo, teremos que usar três 
	\textit{stencils} (os dois restantes podem ser vistos na Tabela 
	\ref{tab:stencils3D}), um para cada dimensão espacial ou temporal. Para tal, 
	podemos utilizar outra analogia: um \textit{array} tridimensional, onde cada
	plano de posições no espaço é um instante no tempo. Tal analogia é representada 
	na Figura \ref{fig:stencil3D}.
	
	\begin{table}[H]
		\centering
		\caption{Demais fórmulas de Diferenças Finitas para as dimensões
			espacial $y$ e temporal $t$}
		\begin{tabular}{c|c}
			\hline
			$\dfrac{\partial^2 u}{\partial y^2}$ & $\dfrac{u_{i,j-1,k} - 2u_{i,j,k} + 
				u_{i,j+1,k}}{\Delta y^2}$ \\ \hline
			$\dfrac{\partial^2 u}{\partial t^2}$ & $\dfrac{u_{i,j,k-1} - 2u_{i,j,k} + 
				u_{i,j,k+1}}{\Delta t^2}$ \\ \hline
		\end{tabular}
		\label{tab:stencils3D}
	\end{table}
	
	%	\begin{figure}[H]
	%		\centering
	%       \label{fig:stencil3D}
	%		\includegraphics[scale=.5, width=\textwidth]{imagefile}
	%		\caption{\textit{Stencil} tridimensional}
	%	\end{figure}
	
	Contudo, até então, não falamos sobre o método de Diferenças Finitas 
	em si. Trata-se de uma sequência de iterações que marcham em função de
	alguma variável. No caso desse trabalho, o avanço se dá no tempo. Essa marcha no tempo quer dizer que o valor para um ponto no espaço no próximo instante de tempo será calculado com base nos valores para pontos no espaço em instantes anteriores. Vamos ser explícitos. Temos que a equação \ref{eq:waveEq} traduzida nas fórmulas vistas nas tabelas 
	\ref{tab:stencil1D} e \ref{tab:stencils3D} se dá por
	\begin{equation}
		 \dfrac{u_{i,j,k-1} - 2u_{i,j,k} + u_{i,j,k+1}}{\Delta t^2} = \dfrac{1}{v^2}
			 \left(\dfrac{u_{i-1,j,k} - 2u_{i,j,k} + \\u_{i+1,j,k}}{\Delta x^2} + 
			 \dfrac{u_{i,j-1,k} - 2u_{i,j,k} + u_{i,j+1,k}}{\Delta y^2}\right) + f(x, y, t)
	\end{equation}
	onde $u_{k+1}$ é o ponto com valor a ser calculado para o próximo
	instante. Logo, ele precisa ser isolado, o que é mostrado na equação 
	\ref{eq:timeMarching}
	\begin{equation}
		\label{eq:timeMarching}
		u_{i, j,k+1} = \dfrac{\Delta t^2}{v^2}
		\left(\dfrac{u_{i-1,j,k} - 2u_{i,j,k} + u_{i+1,j,k}}{\Delta x^2} + 
		\dfrac{u_{i,j-1k} - 2u_{i,j,k} + u_{i,j+1,k}}{\Delta y^2}\right) + \Delta t^2f(x, y, t) - u_{i,j,k-1} + 2u_{i,j,k}
	\end{equation}
		\todo[inline]{buscar referências sobre o método na PIC01}
