\section{O Método de Diferenças Finitas (MDF)}

	Uma equação diferencial parcial, que geralmente é considerada em um domínio 
	contínuo, pode ser discretizada. Isso é feito para que a equação possa ser 
	representada e resolvida computacionalmente. 
	
	No caso do método de diferenças	finitas para esse trabalho, basta 
	transcrever cada termo da equação para o equivalente na fórmula de 
	diferenças finitas para derivações de segundo grau, que é
	\begin{table}[H]
		\centering
		\begin{tabular}{c|c}
			\hline
			$\dfrac{\partial^2 u}{\partial x^2}$ & $\dfrac{u_{i-1} - 2u_i + u_{i+1}}{2\Delta x^2}$ \\ \hline
		\end{tabular}
	\end{table}
	
