\chapter{Paralelismo em prática}

Para facilitar o desenvolvimento de \textit{software} com paralelização, foram 
criadas interfaces com rotinas (funções) que servem de abstração para o desenvolvedor.
Dessa forma, este se preocupará apenas com as estratégias de paralelização a 
serem adotadas e não como o paralelismo deve ocorrer em \gls{low-level}. Tais interfaces 
são chamadas \acrfull{api} e são indispensáveis para esse trabalho.

Nesse capítulo serão abordadas as três \glspl{API} que foram utilizadas para a 
realização desse trabalho: \acrshort{openmp}, \acrshort{pthreads} e \acrshort{openmpi}.

\section{OpenMP}

\label{sec:openmp}

A \acrshort{api} \acrfull{openmp} consiste em rotinas, 
variáveis de ambiente e diretivas de compilação reunidas em um 
modelo portável e escalável que serve como uma interface para 
desenvolvedores criarem aplicações paralelas com simplicidade e 
flexibilidade \cite{wiki:openmp}. Essa \acrshort{api} se encontra 
incluída no modelo de computação paralela de memória compartilhada, 
assim como a \acrshort{pthreads}.

\subsection{Diretiva Utilizada}
\begin{lstlisting}
#pragma omp parallel for
\end{lstlisting}
Considerada uma diretiva de compartilhamento de trabalho, essa 
ferramenta permite que qualquer laço de repetição do tipo \texttt{for} 
(nas linguagens C/C++, 
\lstinline[columns=fixed]{for(int i = 0; i < ALGUM_NUMERO; i++)// faz algo}, 
por exemplo) tenha seu número de iterações dividido entre $n$ \textit{\gls{threads}}, 
sendo $0 < n <= \text{número máximo de threads}$.

\section{\texttt{Pthreads}}

\section{MPI}

	\begin{itemize}
		\item Do que se trata a MPI e onde ela está incluída;
		\item um programa hello world com MPI.
	\end{itemize}
