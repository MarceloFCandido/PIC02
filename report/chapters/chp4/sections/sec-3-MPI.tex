\section{MPI}

\acrfull{mpi} é uma especificação de biblioteca para comunicação entre \glspl{node} 
através do envio e recebimento de mensagens
(modelo distribuído de passagem de mensagens), sendo que os dados do espaço de
endereçamento %\cite{wiki:endereco} 
de um processo (instanciado pelo \acrshort{mpi})
são passados para o espaço de outro ``através de operações cooperativas em cada
processo'' (tradução nossa) \cite{doc:mpi}.

É importante relatar que a \acrshort{mpi} não é uma implementação em si, mas
sim uma especificação, independente de fornecedores. Com isso, existem
diversas implementações e será utilizada apenas uma delas nesse trabalho:
\hyperlink{https://www.open-mpi.org/}{OpenMPI}.

\subsection{Rotinas Utilizadas}

\subsubsection{MPI\_Init}
\begin{lstlisting}
int MPI_Init(int *argc, char ***argv)
\end{lstlisting}
Rotina responsável por inicializar o ambiente \acrshort{mpi}, sendo que seus 
argumentos são referências aos argumentos do programa principal no qual o ambiente 
é inicializado \cite{man:mpi_init}.
\\
\subsubsection{MPI\_Comm\_size}
\begin{lstlisting}
int MPI_Comm_size ( MPI_Comm comm, int *size )
\end{lstlisting}
A rotina acima retorna o tamanho de um grupo associado ao comunicador passado por
parâmetro \cite{man:mpi_comm_size}. No contexto do padrão \acrshort{mpi}, um comunicador 
é um objeto que descreve um  grupo de processos \cite{victor:openMP-MPI}.
\\
\subsubsection{MPI\_Comm\_rank}
\begin{lstlisting}
int MPI_Comm_rank ( MPI_Comm comm, int *rank )
\end{lstlisting}
Essa rotina retorna a posição do \gls{comp-proc}, que a chama, no \textit{rank} do 
comunicador \cite{man:mpi_comm_rank}.
\\
\subsubsection{MPI\_Send}
\begin{lstlisting}
int MPI_Send( void *buf, int count, MPI_Datatype datatype, int dest,
                     int tag, MPI_Comm comm )
\end{lstlisting}
Rotina que tem como objetivo enviar o conteúdo do \textit{buffer} apontado por 
\texttt{buf}, que contém um número (\texttt{count}) de elementos do tipo dado por 
\texttt{datatype} para o processo destino (\texttt{dest}), numerado conforme o 
\textit{rank} do comunicador \texttt{comm}. O parâmetro \texttt{tag} é ignorado nesse 
trabalho.

Quando o comando dessa rotina é alcançado no código, a execução do mesmo não continua 
até que a mensagem que está sendo enviada seja recebida \cite{man:mpi_send}.
\\
\subsubsection{MPI\_Recv}
\begin{lstlisting}
int MPI_Recv( void *buf, int count, MPI_Datatype datatype, int source,
                     int tag, MPI_Comm comm, MPI_Status *status )
\end{lstlisting}
Essa rotina trabalha semelhantemente à \texttt{MPI\_Send}, mas dessa vez recebe uma mensagem 
de \texttt{src} \cite{man:mpi_recv}. O argumento \texttt{status} é ignorado nesse trabalho.
\\
\subsubsection{MPI\_Finalize}
\begin{lstlisting}
int MPI_Finalize()
\end{lstlisting}
Simplesmente finaliza o ambiente de execução \acrshort{mpi} \cite{man:mpi_finalize}.
