\section{Pthreads}

POSIX \textit{Threads}, abreviada para Pthreads, é uma \acrshort{api} que serve como
um ``modelo de execução paralela'' usando memória compartilhada como a OpenMP,
mas fornecendo rotinas para criação e manipulação de \textit{threads} e outras funcionalidades
fora do escopo desse trabalho \cite{wiki:pthreads, LLNL:pthreads}. Essa \acrshort{api} permite
mais controle sobre o código, dando mais liberdade ao programador em definir a
paralelização deste.

\subsection{Rotinas Utilizadas}

Nessa seção são listadas e brevemente explicadas as rotinas da \acrshort{api}
\acrshort{pthreads} usadas nesse trabalho.

\subsubsection{phtread\_create}
\begin{lstlisting}
int pthread_create(pthread_t *restrict thread, const pthread_attr_t *restrict attr, 
	void *(*start_routine)(void*), void *restrict arg);
\end{lstlisting}
Essa rotina dá inicio a uma \textit{\gls{threads}} (primeiro argumento), podendo definí-la com 
um atributo previamente criado. Tal \textit{\gls{threads}} executará a rotina cuja referência é 
passada como argumento. O argumento \texttt{arg} é o parâmetro passado à rotina referenciada \cite{man:pthread_create}.
\\
\subsubsection{phtread\_exit}
\begin{lstlisting}
void pthread_exit(void *value_ptr);
\end{lstlisting}
Rotina responsável por terminar a execução de uma \textit{\gls{threads}} \cite{man:pthread_exit}.
\\
\subsubsection{pthread\_attr\_init}
\begin{lstlisting}
int pthread_attr_init(pthread_attr_t *attr);
\end{lstlisting}
Rotina responsável por inicializar um atributo a ser utilizado na rotina \texttt{pthread\_create}. 
Tal atributo pode determinar que as \textit{\gls{threads}} a serem criadas com ele sejam 
sincronizáveis, por exemplo, o que será utilizado no capítulo [\ref{cap:implementation}] 
\cite{man:pthread_attr_init}.
\\
\subsubsection{pthread\_setdetachedstate}
\begin{lstlisting}
int pthread_attr_setdetachstate(pthread_attr_t *attr, int detachstate);
\end{lstlisting}
Rotina que possibilita a determinação do atributo \texttt{attr}, a ser passado a 
uma \textit{\gls{threads}} em sua criação, como \textit{joinable}, ou seja, 
sincronizável. Esse atributo também pode ser determinado como \textit{detached}, 
ou seja, desanexado; contudo isso foge ao escopo desse trabalho 
\cite{man:pthread_attr_setdetachedstate}.
\\
\subsubsection{pthread\_join}
\begin{lstlisting}
int pthread_join(pthread_t thread, void **value_ptr);
\end{lstlisting}
Essa rotina é responsável por pausar a execução da \textit{\gls{threads}} que a 
chamou enquanto a \textit{\gls{threads}}-alvo (passada por argumento) não terminar. 
Caso a \textit{\gls{threads}}-alvo já tenha terminado, a função simplesmente retorna 
com sucesso.

Essa função pode ser utilizada pelo desenvolvedor para sincronizar o funcionamento 
de um conjunto de \textit{\glspl{threads}}. Ou seja, ela permite a criação de ``áreas 
paralelas'' no código, cujo início se dá na criação das \textit{\glspl{threads}} e 
o fim quando a dita rotina for chamada para todas as \textit{\glspl{threads}} 
existentes. Dessa forma, o programa não passará da área paralela enquanto houver alguma 
\textit{\gls{threads}} sem terminar \cite{man:pthread_join}.
\\